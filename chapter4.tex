La obtención de la forma aparente de los Radios Característicos de los choques de Proa
no se obtiene de manera analítica de forma sencilla, por lo que recurrimos a hacer
aproximaciones a la forma de un choque dado, utilizando las cuádricas de revolución.
Éstas cuádricas dan un buen ajuste pero no son capaces de reproducir la forma completa
de un choque de proa dado, por lo que recurrimos al uso de dos cuádricas que en conjunto
ajustan a la forma completa del choque: una para la ``cabeza'' del choque, y otro para la cola.
Y cómo ya vimos en la sección , los radios característicos aparentes se pueden obtener de
manera sencilla para estas superficies.

\section{Ajustes a la cabeza}
\section{Ajustes a la cola}