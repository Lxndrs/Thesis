\appendix
% \appendixpage
% \addappheadtotoc
\newcommand{\norm}[1]{\left\lVert#1\right\rVert}
\section{Derivación Matemática del Radio de Curvatura}
\label{app:math-curvature radius}

Tomamos una curva genérica $\vec{\sigma}(t) \equiv x(t) \hat{x} + y(t) \hat{y}$ continua y suave para todo valor de $t$ real y finito.
Sus derivadas se denotan como $\vec{\sigma}'(t)$ y $\vec{\sigma}''(t)$. Su longitud de arco está dada por:

\begin{align}
  s(t) = \int^t_0 \norm{\vec{\sigma}'(t')}~dt' \label{eq:arc-length}
\end{align}

Reparametrizamos la trayectoria $\vec{\sigma}(t)$ con la longitud de arco y diferenciando respecto a ésta obtenemos lo siguiente:

\begin{align}
  \frac{d\vec{\sigma}}{ds} = \frac{\vec{\sigma}'(t)}{\norm{\vec{\sigma}'(t)}} \equiv \vec{T}(s)
\end{align}

Esta última expresión se logró diferenciando la ecuación (\ref{eq:arc-length}) y aplicando la regla de la cadena al diferenciar
$\frac{d\vec{\sigma}}{ds}$. $\vec{T}(s)$ es el vector tangente a la trayectoria $\vec{\sigma}(s)$.

La curvatura $\kappa$ se define como la magnitud de la derivada del vector tangente respecto a la longitud de arco, o bien, como la
segunda derivada de la trayectoria $\vec{\sigma}(s)$:

\begin{align}
  \kappa \equiv \norm{\vec{T}'(s)} = \norm{\vec{\sigma}''(s)} 
\end{align}

El radio de curvatura se define como el radio de un círculo que ajusta localmente a la trayectoria, y se calcula como el inverso
multiplicativo de la curvatura:

\begin{align}
  R_c = \frac{1}{\kappa}
\end{align}

Aplicando la regla de la cadena encontramos la siguiente expresión para la curvatura:

\begin{align}
  \kappa = \norm{\frac{\vec{\sigma}''(t)}{\norm{\vec{\sigma}'(t)}^2} - \frac{\vec{\sigma}'(t)}{\norm{\vec{\sigma}'(t)}^4}
  \vec{\sigma}'(t)\cdot\vec{\sigma}''(t)} \label{eq:curvature}
\end{align}

Escribimos las componentes de las derivadas de $\vec{\sigma}(t)$ para calcular los factores que intervienen en la
ecuación (\ref{eq:curvature}):

\begin{align} 
  \vec{\sigma}'(t) &= \dot{x}\hat{x} + \dot{y} \hat{y} \\
  \vec{\sigma}''(t) &= \ddot{x} \hat{x} +  \ddot{y} \hat{y} \\
  \implies \vec{\sigma}'(t)\cdot\vec{\sigma}''(t) &= \dot{x}\ddot{x} + \dot{y}\ddot{y}
\end{align}

De esta forma, calculamos la curvatura como sigue:

\begin{align}
  \kappa &= \left[\left(\frac{\ddot{x}}{\norm{\vec{\sigma}'}^2} - \frac{\dot{x}(\dot{x}\ddot{x} + \dot{y}\ddot{y})}{\norm{\vec{\sigma}'}^4}\right)^2
  + \left(\frac{\ddot{y}}{\norm{\vec{\sigma}'}^2} - \frac{\dot{y}(\dot{x}\ddot{x} + \dot{y}\ddot{y})}{\norm{\vec{\sigma}'}^4} 
\right)^2\right]^{1/2} \\
 \kappa  &= \norm{\vec{\sigma}'}^{-2}\left[\left(\ddot{x}(\dot{x}^2 + \dot{y}^2) - \dot{x}(\dot{x}\ddot{x} + \dot{y}\ddot{y})\right)^2
     + \left(\ddot{y}(\dot{x}^2 + \dot{y}^2) - \dot{y}(\dot{x}\ddot{x} + \dot{y}\ddot{y})\right)^2 \right]^{1/2}\\
 \kappa  &= \norm{\vec{\sigma}'}^{-2}\left[\left(\ddot{x}\dot{y}^2 - \dot{x}\dot{y}\ddot{y}\right)^2 +
   \left(\ddot{y}\dot{x}^2 - \dot{y}\dot{x}\ddot{x}\right)^2 \right]^{1/2} \\
 \kappa  &=\norm{\vec{\sigma}'}^{-3}\left[\ddot{x}^2\dot{y}^2 + \ddot{y}^2\dot{x}^2 - 2\ddot{x}\ddot{y}\dot{x}\dot{y}\right]^{1/2} \\
\kappa &= \frac{\left|\ddot{x}\dot{y} - \ddot{y}\dot{x}\right|}{\left(\dot{x}^2+\dot{y}^2\right)^{3/2}} 
\end{align}

Utilizando coordenadas polares, la expresión para el radio de curvatura queda como sigue:

\begin{align}
R_c = R\frac{\left(1 + \omega^2\right)^{3/2}}{\left|1 + \omega^2 - \dot{\omega}\right|}\label{eq:Rc-generic}
\end{align}

Donde $\omega \equiv \frac{1}{R}\frac{dR}{d\theta}$ (ver sección \ref{sec:projection})

\subsection{Radio de curvatura para un polinomio de grado $2n$}
\label{app:curvature-radius-poly}

Dado que es de nuestro interés calcular el radio de curvatura del choque en el eje de simetría
$(\theta=0)$, debido a que es analíticamente más fácil de calcular, y a su vez es medible
observacionalmente ajustando un círculo a una serie de mediciones de la posición del choque (sección ).
Entonces, hacemos una aproximación para la función $R(\theta)$ que nos da la forma del choque
mediante un polinomio par de grado $2n$ de la siguiente forma:

\begin{align}
R(\theta) \simeq R_0\left(1 + \gamma\theta^2 + \Gamma\theta^4 + \cdots + \Gamma_n\theta^n\right)
\end{align}

De esta forma calculamos las derivadas de $R(\theta)$:

\begin{align}
  \dot{R}(\theta) &\simeq R_0\theta\left(2\gamma + 4\Gamma\theta^2 + \cdots + n\Gamma_n\theta^{n-2}\right) \\
  \ddot{R}(\theta) &\simeq R_0\left(2\gamma + 12\Gamma\theta^2 + \cdots + n(n-1)\Gamma_n\theta^{n-2}\right)
\end{align}

Evaluando en $\theta = 0$ obtenemos los siguiente:

\begin{align}
  R(0) &= R_0 \\
  \dot{R}(0) &= 0 \\
  \ddot{R}(0) &= 2\gamma R_0 \\
  \implies \omega(0) &= 0 \\
  \dot{omega}(0) &\equiv \frac{\ddot{R}(0)}{R(0)} - \left(\frac{\dot{R}(0)}{R(0)}\right)^2 = 2\gamma
\end{align}

Sustituyendo en la ecuación (\ref{eq:Rc-generic}) obtenemos el radio de curvatura en $\theta=0$:

\begin{align}
  R_c = \frac{R_0}{\left|1 - 2\gamma\right|}
\end{align}

Por esto en el apéndice \ref{app:dervation-radii} para encontrar el radio de curvatura nos enfocamos en encontrar
el coeficiente de segundo orden en la expansión en serie de $R(\theta)$.

\section{Derivación paso a paso de los Radios Característicos en la aproximación
  hipersónica del modelo de interacción de dos vientos}
\label{app:dervation-radii}

\subsection{$R_0$}
Podemos determinar el radio característico $R_0$ a partir de la condición de que el choque es estacionario. En este caso,
los momentos de los dos vientos son iguales en la posición del choque. Por tanto, utilizando la ecuación de momento en $\theta=0$
obtenemos lo siguiente:

\begin{align}
  \rho_{ws} v^2_w &= \rho_{ws1} v^2_{w1}
\end{align}
Donde $\rho_{ws}$ y $\rho_{ws1}$ son las densidades de los dos vientos en la posición del choque. Por otro lado, como la tasa de pérdida
de masa es constante para un ángulo $\theta$ dado, entonces podemos hacer lo siguiente:

\begin{align}
  \frac{\dot{M}^0_w}{4\pi R^2 v_w}v^2_w &= \frac{\dot{M}^0_{w1}}{4\pi\left(D-R\right)^2v_{w1}}v^2_{w1} \label{eq:momento-R0-uf}
\end{align}
En esta última ecuación hemos sustituído $\dot{M}^0_w = 4\pi R_0^2 v_w \rho_{ws}$ y \\
$\dot{M}^0_{w1} = 4\pi \left(D - R_0\right)^2 v_{w1} \rho_{ws1}$ aprovechando que dichas cantidades deben conservarse.

Reduciendo la ecuación (\ref{eq:momento-R0-uf}) encontramos una expresión para $R_0$:

\begin{align}
  \frac{R_0}{D} = \frac{\beta^{1/2}}{1+\beta^{1/2}}
\end{align}

\subsection{$\tilde{R}_{90}$}
$R_{90}$ puede determinarse a partir de evaluar las ecuaciones (\ref{eq:R-geometric}) y (\ref{eq:th1-th}) en $\theta=\frac{\pi}{2}$
como sigue:

\begin{align}
  \frac{R_{90}}{D} &= \tan\theta_{1,90} \\
  \theta_{1,90}\cot\theta_{1,90} -1 &=  - \frac{2\beta}{k+2}
\end{align}

Donde $\theta_{1,90} = \theta_1\left(\frac{\pi}{2}\right)$. Introducimos un nuevo parámetro $\xi \equiv \frac{2}{k+2}$ de modo que
combinando las dos ecuaciones anteriores obtenemos lo siguiente:

\begin{align}
  \frac{R_{90}}{D} = \tan\theta_{1,90} = \frac{\theta_{1,90}}{1-\xi\beta} \label{eq:r90-incomplete}
\end{align}

Hacemos una expansión en serie para el lado izquierdo de la ecuación y reducimos:

\begin{align}
  \theta^2_{1,90}\left(1 + \frac{\theta^2_{1,90}}{15}\right) \simeq 3\beta\xi
\end{align}
Tomamos la solución a primer orden $\theta_{1,90} = 3\beta\xi$,  sustituímos este valor en el término correctivo y resolvemos para
$\theta_{1,90}$:

\begin{align}
  \theta_{1,90} = \left(\frac{3\xi\beta}{1+\frac{1}{5}\xi\beta}\right)^{1/2} \label{eq:th190}
\end{align}

Finalmente sustituímos (\ref{eq:th190}) en (\ref{eq:r90-incomplete}) para obtener $R_{90}$:
\begin{align}
  \frac{R_{90}}{D} &= \frac{\left(3\xi\beta\right)^{1/2}}{\left(1+\frac{1}{5}\xi\beta\right)^{1/2}\left(1-\xi\beta\right)} \\
  \tilde{R}_{90} &= \frac{\left(3\xi\right)^{1/2}\left(1+\beta^{1/2}\right)}
                   {\left(1+\frac{1}{5}\xi\beta\right)^{1/2}\left(1-\xi\beta\right)} 
\end{align}

Siendo que el choque de proa en nuestro modelo genérico es simétrico, entonces la forma $R(\theta)$ debe ser una función par,
por tanto podemos hacer la siguiente expansión en serie:

\begin{align}
  R(\theta) \simeq R_0\left(1 + \gamma\theta^2 + \Gamma\theta^4 +\cdots +\cdots \right)
\end{align}

De esta forma el radio de curvatura para $\theta=0$ queda como sigue (ver apéndice \ref{app:math-curvature radius}):

\begin{align}
  \tilde{R}_c = \left(1 - 2\gamma\right)^{-1}
\end{align}

Para encontrar el coeficiente de segundo orden $\gamma$ hacemos una expansión en serie de las ecuaciones (\ref{eq:th1-th}) y
(\ref{eq:R-geometric}) para ángulos pequeños, mostrando a continuación la expansión de cada término para al final hacer la
reducción algebraica:

\begin{align}
  \theta_1\cot\theta_1 -1 &\simeq -\frac{\theta^2_1}{3}\left(1 + \frac{\theta^2_1}{15}\right) \\
  \cos^k\theta &\simeq \left(1 - \frac{\theta^2}{2}\right)^k \simeq \left(1 - \frac{k\theta^2}{2}\right) \\
  \sin^2\theta &\simeq \theta^2\left(1 - \frac{\theta^2}{6}\right)^2 \simeq \theta^2\left(1 - \frac{\theta^2}{3}\right)\\
  \implies \cos^k\theta\sin\theta &\simeq \theta^2 - \left(\frac{1}{3} + \frac{k}{2}\right) \\
  \implies I_k(\theta) &\simeq \frac{\theta^3}{3}\left[1 - \frac{1}{10}\left(3k + 2\right)\theta^2\right] \\
  \cot\theta &\simeq \theta^{-1}\left(1 - \frac{\theta^2}{3}\right) \\
  \implies 2\beta I_k(\theta)\cot\theta &\simeq \frac{2}{3}\beta\theta^2\left[1 - \frac{1}{30}\left(9k + 16\right)\theta^2\right] \\
  -\frac{2\beta}{k+2}\left(1 - \cos^{k+2}\theta\right) &\simeq \beta\theta^2\left[1 - \frac{1}{12}\left(3k+4\right)\right] \\
\end{align}

Sustituyendo las expansiones anteriores en (\ref{eq:th1-th}) obtenemos lo siguiente:
\begin{align}
  \theta^2_1\left(1 + \frac{\theta^2_1}{15}\right) \simeq \beta\theta^2\left[1 + \frac{1}{60}\left(4 - 9k\right)\theta^2\right]
  \label{eq:th1-th-approx}
\end{align}

La solución a primer orden (ignorando el término cuártico) es $\theta^2_ 1 \simeq \beta\theta^2$. Sustituímos esta solución en el
término correctivo y resolvemos para $\theta^2_1$:

\begin{align}
  \theta^2_1 &\simeq \beta\theta^2\left[1 + \frac{1}{60}\left(4 - 9k\right)\theta^2\right]\left(1 + \frac{\beta\theta^2}{15}\right)^{-1} \\
  \theta^2_1 &\simeq \beta\theta^2\left[1 + \frac{1}{60}\left(4 - 9k\right)\theta^2\right]\left(1 - \frac{\beta\theta^2}{15}\right) \\
  \implies \theta^2_1 &\simeq \beta\theta^2\left(1 + 2C_{k\beta}\theta^2\right) \\
  \mathrm{Donde:\quad}C_{k\beta} &= \frac{1}{2}\left(A_k - \frac{\beta}{15}\right) \\
  A_k &= \frac{1}{15} - \frac{3k}{20}
\end{align}

Utilizamos esta solución para $\theta_1$ en la ecuación (\ref{eq:R-geometric}), ignorando términos de orden superior al cuártico
dentro de los corchetes:

\begin{align}
  \theta_1 &\simeq \beta^{1/2}\theta\left(1 + 2C_{k\beta}\theta^2\right)^{1/2} \\
  \implies \theta + \theta_1 &\simeq \theta\left[1 + \beta^{1/2}\left(1 + 2C_{k\beta}\theta^2\right)^{1/2}\right]
  \simeq \theta\left[1 + \beta^{1/2}\left(1 + C_{k\beta}\theta^2\right)\right]\\ 
  \sin\theta_1 &\simeq \theta_1\left(1 - \frac{\theta^2_1}{6}\right) \\
  &\simeq \beta^{1/2}\theta\left(1 + 2C_{k\beta}\theta^2\right)^{1/2}\left[1 - \frac{\beta\theta^2\left(1 + 2C_{k\beta}\theta^2\right)}{6}\right]\\
  &\simeq \beta^{1/2}\theta\left(1 + C_{k\beta}\theta^2\right)\left[1 - \frac{\beta\theta^2\left(1 + 2C_{k\beta}\theta^2\right)}{6}\right]\\
           &\simeq \beta^{1/2}\theta\left[\left(1 + C_{k\beta}\theta^2\right) - \frac{\beta\theta^2\left(1 + 2C_{k\beta}\theta^2\right)
             \left(1 + C_{k\beta}\theta^2\right)}{6}\right]\\
  \implies \sin\theta_1 &\simeq \beta^{1/2}\theta\left[1 + \left(C_{k\beta}- \frac{\beta}{6}\right)\theta^2\right] \\
  \sin\left(\theta + \theta_1 \right) &\simeq \left(\theta + \theta_1\right)\left[1 - \frac{\left(\theta + \theta_1\right)^2}{6}\right] \\
           &\simeq \theta\left[1 + \beta^{1/2}\left(1 + C_{k\beta}\theta^2\right)\right] \left[1 - \frac{\theta^2\left(1 + \beta^{1/2}
             \left(1 + C_{k\beta}\theta^2\right)\right)^2}{6}\right] \\
           &\simeq \theta\left[1 + \beta^{1/2}\left(1+ C_{k\beta}\theta^2\right)\right]\left[1 - \frac{\theta^2}{6}\left(1 + \beta^{1/2}\right)^2
             \left(1 + \frac{2\beta}{1 + \beta^{1/2}}C_{k\beta}\theta^2\right)\right] \\
           &\simeq \theta\left[1 + \beta^{1/2} - \frac{1}{6}\left(1+\beta^{1/2}\right)^3\theta^2 + \beta^{1/2}C_{k\beta}\theta^2\right] \\
           &\simeq \theta\left(1 + \beta^{1/2}\right)\left[1 + \left(\frac{\beta^{1/2}C_{k\beta}}{1 + \beta^{1/2}} - \frac{1}{6}\left(1 + \beta^{1/2}\right)^2
             \theta^2\right)\right]\\
  \implies \frac{R}{D} &\simeq \frac{\beta^{1/2}}{1 + \beta^{1/2}}\left[1 + \left(C_{k\beta} - \frac{\beta}{6}\right)\theta^2\right]
                         \left[1 + \left(\frac{\beta^{1/2}C_{k\beta}}{1 + \beta^{1/2}} - \frac{\left(1 + \beta^{1/2}\right)^2}{6}\right)\right]^{-1} \\
  &\simeq \frac{\beta^{1/2}}{1 + \beta^{1/2}}\left[1 + \left(C_{k\beta} - \frac{\beta}{6}\right)\theta^2\right]
    \left[1 - \left(\frac{\beta^{1/2}C_{k\beta}}{1 + \beta^{1/2}} - \frac{\left(1 - \beta^{1/2}\right)^2}{6}\right)\right] \\
           &\simeq R_0\left[1 + \left(\right)\theta^2\right] \\
  &\simeq R_0 \left[1 + \left(\right)\theta^2\right] 
\end{align}
