\section{Vientos Estelares}
\section{Choques}
\section{Frentes de Ionización}
Un frente de ionización es la interfaz entre un medio gaseoso neutro y uno
ionizado. Ocurren cerca de fuentes de rediación ionizante, tales como
estrellas masivas, tipo B temprana o tipo O. el frente de ionización puede
tratarse como una discontinuidad en el medio gaseoso

\section{Regiones HII}
Las regiones HII se forman cuando una estrella masiva, de tipo espectral
O ó B temprana, ioniza el gas que se encuentra a su alrededor. El gas
ionizado se encuentra en equilibrio térmico, a una temperatura del
orden de $10^4~K$. El principal proceso de calentamiento es la
radiación de la estrella central, mientras que el enfriamiento se da
principalmente por la recombinación de líneas prohibidas y por emisión
libre-libre.

\section{Modelo Genérico de los Choques de Proa}
\label{sec:Modelo-generico}
Para este trabajo consideramos en general dos modelos de
interacción  de vientos:
\begin{itemize}
\item Una fuente localizada en el origen que emite un viento esférico
  que puede ser isotrópico o anisotrópico (figura
  \ref{fig:isotropic-aniso}) no acelerado que interactúa con el viento
  esférico isotrópico de otra fuente que se encuentra a una distancia
  $D$ de la primera(figura \ref{fig:crw-esquema})
\item Una fuente localizada en el origen que emite un viento esférico
  isotrópico no acelerado que interactúa con un viento plano paralelo
  no acelerado y densidad constante (figura )
\end{itemize}
El sitema en su conjunto tiene simtería cilíndrica.
\begin{figure}
  \includegraphics[width=0.7\linewidth]{./Figures/anisotropic-arrows}
  \label{fig:isotropic-aniso}
  \caption{Representación esquemática de vientos con diferentes
    anisotropías:
    Arriba izquierda: Viento isotrópico esférico. Arriba derecha: viento
    isotrópico hemisférico. Abajo: Vientos anisotrópicos donde el
    parámetro $k$ indica el grado de anisotropía (ver sección
    \ref{sec:hipersonica})}
\end{figure}
\begin{figure}
  \includegraphics[width=0.5\linewidth]{./Figures/bowshock-crw-variables}
  \label{fig:crw-esquema}
  \caption{Representación esquemática del problema de interacción de dos vientos:
    Dos fuentes separadas por una distancia $D$ emiten un viento radial que forma un
    choque de proa a una distancia $R$ del origen. El sistema tiene geometría cilíndrica
    siendo el eje $z$ el eje de simetría. La forma del choque depende únicamente del ángulo
    polar $\theta$, medido a partir del origen. Otro ángulo que es de utilidad es $\theta_1$,
  que corresponde al ángulo polar medido a partir de la posición de la otra fuente.}
\end{figure}

\subsection{Radios ``Característicos''}
\label{sec:char-rad}
Las cantidades medibles que nos ayudan a caracterizar un choque de proa las
llamamos ``Radios característicos'' (ilustrados en la figura
\ref{fig:char-radii}):
\begin{itemize}
\item Radio del choque en la dirección del eje de simetría del sistema.
  Denotado como $R_0$
\item Radio en dirección perpendicular al eje de simetría del sistema.
  Denotado como $R_{90}$
\item Radio de curvatura en la ``nariz'' del choque de proa. Denotado
  como$ R_c$
\end{itemize}

\begin{figure}
  \includegraphics[width=\linewidth]{./Figures/characteristic-radii}
  \label{fig:char-radii}
  \caption{Representación esquemática de los radios característicos
  de un choque de proa}
\end{figure}.
Para este trabajo resulta útil hacer una noramlización de los radios
característicos u otros radios, para que las mediciones que obtengamos
sean adimensionales. De esta forma, podemos hacer la normalización con
la distancia $D$, o bien con $R_0$, dependiendo de qué tipo de
normalización resulte más conveniente. En el primer caso expresamos
explícitamente el cociente (e.g $\frac{R_0}{D}$, $\frac{R_c}{D}$,
$\frac{R_{90}}{D}$), y en el segundo caso añadiremos una tilde al
radio en cuestión (e.g $\tilde{R}_c$, $\tilde{R}_{90}$). 

\section[Aproximación Hipersónica]{Interacción de dos vientos: Aproximación Hipersónica \citep{Canto:1996}}
\label{sec:hipersonica}

El problema de interacción de dos vientos es de gran interés en astrofísica, y
ha sido estudiado en múltiples ocasiones, principalmente mediante simulaciones
hidrodinámicas. Sin embargo, cuando se toman en cuenta diversos factores, incluídos
conservación de masa, momento y momento angular, el problema puede resolverse de manera
algebraica.
\subsection{Cantidades conservadas en un flujo hipersónico de capa delgada}

Consideramos dos flujos hipersónicos, no acelerados que forman una capa estacionaria delgada
formada por dos choques radiativos separados por una discontinuidad de contacto. El sistema
tiene geometría cilíndrica y los vientos no tienen velocidad azimutal. Bajo estos términos,
describimos la posición de la capa delgada como $R(\theta)$, donde $R$ es el radio de la capa
medido a partir de la posición del origen del viento con menor momento y $\theta$ es el ángulo
polar. Si asumimos que el gas chocado está bien mezclado, entonces tiene una sola velocidad
pos choque dada por:

\begin{align}
  \vec{v} = v_r \hat{r} + v_z \hat{z}
\end{align}

Donde el eje de simetría del sistema es paralelo a $\hat{z}$, y $\hat{r}$ es el radio cilíndrico.
Definimos $\dot{M}(\theta)$, $\vec{\dot{\Pi}}(\theta)$ y $\dot{J}(\theta)$ como la tasa de pérdida
de masa, la tasa de momento y la tasa de momento angular, respectivamente, de la capa delgada
integradas desde $\theta=0$ hasta $\theta$. Éstas se calculan de la siguiente manera:

\begin{align}
  \vec{\dot{\Pi}}(\theta) &= \dot{\Pi}_r(\theta) \hat{r} + \dot{\Pi}_z(\theta) \hat{z} = \dot{M}\left(
                      v_r \hat{r} + v_z\hat{z}\right) \label{eq:dot-pi}\\
  \vec{\dot{J}}(\theta) &= \vec{R}(\theta) \times \vec{\dot{\Pi}}(\theta)  \\
  \dot{M}(\theta) &= \dot{M}_w(\theta) + \dot{M}_{w1} \label{eq:dot-M}
\end{align}
Donde $\vec{R}(\theta)\equiv R(\theta)\sin\theta \hat{r} + R\cos\theta \hat{z}$. Resolviendo el producto
cruz y tomando su magnitud encontramos que:

\begin{align}
  \dot{J}(\theta) &= \dot{M}(\theta)R(\theta)v_\theta \label{eq:dot-J}\\
  \mathrm{donde:~} & v_\theta = v_r\cos\theta - v_z\sin\theta \label{eq:v-theta}
\end{align}

Por otro lado, al asumir estado estacionario, necesitamos que la tasa de pérdida de masa, la tasa de
momento y la tasa de momento angular de la capa delgada sean iguales a aquellas inyectadas por los dos
vientos. Entonces definimos estas cantidades como $\dot{M}_w$, $\dot{\Pi}_{wr}$, $\dot{\Pi}_{wz}$ y
$\dot{J}_{w}$ para el viento con menor momento, y para el otro viento se utiliza la misma notación
solo que utilizando el subíndice ``w1''. De esta forma tenemos que:
\begin{align}
  \dot{\Pi}_r(\theta)\hat{r} + \dot{\Pi}_z(\theta)\hat{z} &= \left[\dot{\Pi}_{wr}(\theta)+ \dot{\Pi}_{wr1}(\theta)
                                                            \right]\hat{r} + \left[\dot{\Pi}_{wz}(\theta)+ \dot{\Pi}_{wz1}(\theta)\right]\hat{z}
                                                            \label{eq:Pi-2} \\
  \dot{J} &=\dot{J}_w(\theta) + \dot{J}_{w1}(\theta) \label{eq:J-2}\\
  \dot{M}(\theta) &= \dot{M}_w(\theta) + \dot{M}_{w1}(\theta) \label{eq:M-2}
\end{align}

Combinando las ecuaciones (\ref{eq:dot-pi}), (\ref{eq:dot-J}), (\ref{eq:dot-M}), (\ref{eq:Pi-2}), (\ref{eq:J-2}) y (\ref{eq:M-2})
encontramos que:

\begin{align}
  \dot{M}(\theta)\left[v_r \hat{r} + v_z\hat{z}\right] &= \left(\dot{\Pi}_{wr}(\theta) + \dot{\Pi}_{wr1}(\theta)\right)\hat{r} +
                                                         \left(\dot{\Pi}_{wz}(\theta) + \dot{\Pi}_{wz1}(\theta)\right)\hat{z} \\
  \dot{M}(\theta)v_\theta R(\theta) &= \dot{J}_w(\theta) + \dot(J)_{w1}(\theta)
\end{align}
Y finalmente combinando con la ecuación (\ref{eq:v-theta}) resolvemos para $R(\theta)$:
\begin{align}
  R(\theta) = \frac{\dot{J}_w(\theta) + \dot(J)_{w1}(\theta)}{\left(\dot{\Pi}_{wr}(\theta) + \dot{\Pi}_{wr1}(\theta)\right)\cos\theta
  - \left(\dot{\Pi}_{wz}(\theta) + \dot{\Pi}_{wz1}(\theta)\right)\sin\theta} \label{eq:R-wind}
\end{align}



\subsection{Problema de Interacción de Dos Vientos}

Aplicamos el formalismo ya mencionado para la interacción de dos vientos radiales. El viento con menor momento
se localiza en el origen, y su densidad a radio fijo varía con el ángulo polar como una ley de potencias
(figura \ref{fig:isotropic-aniso}):
\begin{align}
  n(\theta) = n_0\cos^k\theta \label{eq:anisotropic-density}
\end{align}
Donde el índice $k$ indica el grado de anisotropía del viento ``interno''. Algunos casos particularmente interesantes
son el viento para un proplyd \citep{HA:1998}, donde $(k=1/2)$ y el caso ``isotrópico'' \citep{Canto:1996} donde $k=0$.
Por el momento restringimos al viento ``externo'' como isotrópico. El problema se muestra de manera esquemática en la
figura \ref{fig:crw-esquema}.

Utilizando la ecuación (\ref{eq:anisotropic-density}) encontramos que la tasa de pérdida de masa está dada por:

\begin{align}
  \dot{M}_w = \int^\theta_0\int^{2\pi}_0\rho_w v_w~d\theta~d\phi =
  \frac{M^0_w}{2\left(k+1\right)}\left(1 - \cos^{k+1}\theta\right)
\end{align}
Donde $v_w$ es la velocidad del viento inteno, $\rho_w = n\bar{m}$  es su densidad, $n$ se obtiene de la ecuación
(\ref{eq:anisotropic-density}), $M^0_w = 4\pi r^2_0v_w n_0 \bar{m}$ es la tasa de pérdida de masa integrada hasta
$\theta = \pi$ para el caso isotrópico, $\bar{m}$  es la masa promedio de las partículas del viento y
$r_0$ es el radio del viento al cual se alcanza la velocidad terminal $v_w$. Para un proplyd consideramos que dicho
radio es el del frente de ionización.

Con esto, obtenemos las tasas de momento y momento angular:
\begin{align}
  \dot{\Pi}_{wz} &= \int^\theta_0 v_w\cos\theta~d\dot{M}_w =
                   \frac{v_w \dot{M}^0_w}{2\left(k+2\right)}\left(1 - \cos^{k+2}\theta\right) \label{eq:Pi-wz} \\
  \dot{\Pi}_{wr} &= \int^\theta_0 v_w\sin\theta~d\dot{M}_w = \frac{1}{2}\dot{M}^0_w v_w I_k(\theta) \\
  \dot{J}_w &= \int^\theta_0 |\vec{R} \times \vec{v}_w|d\dot{M}_w = 0
\end{align}

Donde la integral $I_k(\theta) = \int^\theta_0 \cos^k\theta \sin^2\theta~d\theta$ tiene solución analítica para $k=0$,
es una integral elíptica de segundo tipo cuando $k=\frac{1}{2}$ y su solución es aun más compleja para el resto de los
casos. Las tasa de momento angular para el viento interior es cero debido a que éste se mide respecto al origen, donde
se localiza la fuente con menor momento. En este punto los vectores de posición y velocidad para un valor de $\theta$
dado son paralelos.

Para el viento exterior consideramos dos casos principales: un viento esférico e isotrópico y un viento
plano--paralelo de densidad y velocidad constante.

\subsubsection{Interacción con un viento esférico isotrópico}

En este caso tomamos como variable independiente al ángulo polar medido a partir de la posición de la fuente del viento
externo, denotado por $\theta_1$. De esta forma las tasas de pérdida de masa, momento y momento angular quedan como sigue:

\begin{align}
  \dot{M}_{w1} &= \frac{M^0_{w1}}{2}\left(1 - \cos\theta_1\right)\\
  \dot{\Pi}_{wz1} &= -\frac{v_{w1}\dot{M}^0_{w1}}{4}\sin^2\theta_1\\
  \dot{\Pi}_{wr1} &= \frac{v_{w1}\dot{M}^0_{w1}}{4}\left(\theta_1 - \sin\theta_1\cos\theta_1\right)\\
  \dot{J}_{w1} &= \int^{\theta_1}_0 R(\theta)v_{w1}\sin(\pi-\theta-\theta_1)~d\dot{M}_{w1} \label{eq:J1}
\end{align}

Utilizando la ley de los senos (ver figura \ref{fig:crw-esquema}), ecuación (\ref{eq:J1}) queda como sigue:

\begin{align}
  \dot{J}_{w1} &= Dv_{w1}\int^{\theta_1}_0 \sin\theta_1~d\dot{M}_{w1} =
                 \frac{v_{w1}\dot{M}^0_{w1}}{4}\left(\theta_1 - \sin\theta_1\cos\theta_1\right) D \label{eq:J1-iso}
\end{align}

Por otro lado, de la figura \ref{fig:crw-esquema}, podemos deducir la siguiente relación geométrica entre $R(\theta)$,
$\theta$ y $\theta_1$:
\begin{align}
  \frac{R(\theta)}{D} &= \frac{\sin\theta_1}{\sin(\theta+\theta_1)} \label{eq:R-geometric}
\end{align}

Combinando las ecuaciones (\ref{eq:R-wind}), (\ref{eq:Pi-wz}) - (\ref{eq:J1-iso}) y (\ref{eq:R-geometric}) obtenemos una ecuación
implícita que nos indica la dependencia de $\theta_1$ con $\theta$:

\begin{align}
  \theta_1\cot\theta_1 -1 = 2\beta I_k(\theta)\cot\theta - \frac{2\beta}{k+2}\left(1 - \cos^{k+2}\theta\right) \label{eq:th1-th} 
\end{align}
Donde $\beta = \frac{\dot{M}^0_w v_w}{\dot{M}^0_{w1}v_{w1}}$ es el cociente del momentos entre los vientos. Este parámetro, junto con el
índice de anisotropía $k$ son los que determinan la forma del choque de proa. 

En este caso, podemos determinar el radio característico $R_0$ a partir de la condición de que el choque es estacionario. En este caso,
los momentos de los dos vientos son iguales en la posición del choque. Por tanto, utilizando la ecuación de momento en $\theta=0$
obtenemos lo siguiente:

\begin{align}
  \rho_{ws} v^2_w &= \rho_{ws1} v^2_{w1}
\end{align}
Donde $\rho_{ws}$ y $\rho_{ws1}$ son las densidades de los dos vientos en la posición del choque. Por otro lado, como la tasa de pérdida
de masa es constante para un ángulo $\theta$ dado, entonces podemos hacer lo siguiente:

\begin{align}
  \frac{\dot{M}^0_w}{4\pi R^2 v_w}v^2_w &= \frac{\dot{M}^0_{w1}}{4\pi\left(D-R\right)^2v_{w1}}v^2_{w1} \label{eq:momento-R0-uf}
\end{align}
En esta última ecuación hemos sustituído $\dot{M}^0_w = 4\pi R_0^2 v_w \rho_{ws}$ y \\
$\dot{M}^0_{w1} = 4\pi \left(D - R_0\right)^2 v_{w1} \rho_{ws1}$ aprovechando que dichas cantidades deben conservarse.

Reduciendo la ecuación (\ref{eq:momento-R0-uf}) encontramos una expresión para $R_0$:

\begin{align}
  \frac{R_0}{D} = \frac{\beta^{1/2}}{1+\beta^{1/2}}
\end{align}


Los radios característicos $R_{90}$ y $R_c$ pueden determinarse a partir de la ecuación (\ref{eq:th1-th}). El primero evaluando
las ecuaciones (\ref{eq:R-geometric}) y (\ref{eq:th1-th}) en $\theta=\frac{\pi}{2}$ como sigue:

\begin{align}
  \frac{R_{90}}{D} &= \tan\theta_{1,90} \\
  \theta_{1,90}\cot\theta_{1,90} -1 &=  - \frac{2\beta}{k+2} _
\end{align}

Donde $\theta_{1,90} = \theta_1\left(\frac{\pi}{2}\right)$. Introducimos un nuevo parámetro $\xi \equiv \frac{2}{k+2}$ de modo que
combinando las dos ecuaciones anteriores obtenemos lo siguiente:

\begin{align}
  \frac{R_{90}}{D} = \tan\theta_{1,90} = \frac{\theta_{1,90}}{1-\xi\beta} \label{eq:r90-incomplete}
\end{align}

Hacemos una expansión en serie para el lado izquierdo de la ecuación y reducimos:

\begin{align}
  \theta^2_{1,90}\left(1 + \frac{\theta^2_{1,90}}{15}\right) \simeq 3\beta\xi
\end{align}
Tomamos la solución a primer orden $\theta_{1,90} = 3\beta\xi$,  sustituímos este valor en el término correctivo y resolvemos para
$\theta_{1,90}$:

\begin{align}
  \theta_{1,90} = \left(\frac{3\xi\beta}{1+\frac{1}{5}\xi\beta}\right)^{1/2} \label{eq:th190}
\end{align}

Finalmente sustituímos (\ref{eq:th190}) en (\ref{eq:r90-incomplete}) para obtener $R_{90}$:
\begin{align}
  \frac{R_{90}}{D} &= \frac{\left(3\xi\beta\right)^{1/2}}{\left(1+\frac{1}{5}\xi\beta\right)^{1/2}\left(1-\xi\beta\right)} \\
  \tilde{R}_{90} &= \frac{\left(3\xi\right)^{1/2}\left(1+\beta^{1/2}\right)}
                   {\left(1+\frac{1}{5}\xi\beta\right)^{1/2}\left(1-\xi\beta\right)} 
\end{align}

Siendo que el choque de proa en nuestro modelo genérico es simétrico, entonces la forma $R(\theta)$ debe ser una función par,
por tanto podemos hacer la siguiente expansión en serie:

\begin{align}
  R(\theta) \simeq R_0\left(1 + \gamma\theta^2 + \Gamma\theta^4 +\cdots +\cdots \right)
\end{align}

De esta forma el radio de curvatura para $\theta=0$ queda como sigue (ver apéndice ):

\begin{align}
  R_c = R_0\left(1 - 2\gamma\right)^{-1}
\end{align}

Para encontrar el coeficiente de segundo orden $\gamma$ hacemos una expansión en serie de las ecuaciones (\ref{eq:th1-th}) y
(\ref{eq:R-geometric}) para ángulos pequeños, mostrando a continuación la expansión de cada término para al final hacer la
reducción algebraica:

\begin{align}
  \theta_1\cot\theta_1 -1 &\simeq -\frac{\theta^2_1}{3}\left(1 + \frac{\theta^2_1}{15}\right) \\
  \cos^k\theta &\simeq \left(1 - \frac{\theta^2}{2}\right)^k \simeq \left(1 - \frac{k\theta^2}{2}\right) \\
  \sin^2\theta &\simeq \theta^2\left(1 - \frac{\theta^2}{6}\right)^2 \simeq \theta^2\left(1 - \frac{\theta^2}{3}\right)\\
  \implies \cos^k\theta\sin\theta &\simeq \theta^2 - \left(\frac{1}{3} + \frac{k}{2}\right) \\
  \implies I_k(\theta) &\simeq \frac{\theta^3}{3}\left[1 - \frac{1}{10}\left(3k + 2\right)\theta^2\right] \\
  \cot\theta &\simeq \theta^{-1}\left(1 - \frac{\theta^2}{3}\right) \\
  \implies 2\beta I_k(\theta)\cot\theta &\simeq \frac{2}{3}\beta\theta^2\left[1 - \frac{1}{30}\left(9k + 16\right)\theta^2\right] \\
  -\frac{2\beta}{k+2}\left(1 - \cos^{k+2}\theta\right) &\simeq \beta\theta^2\left[1 - \frac{1}{12}\left(3k+4\right)\right] \\
\end{align}

Sustituyendo las expansiones anteriores en (\ref{eq:th1-th}) obtenemos lo siguiente:
\begin{align}
  \theta^2_1\left(1 + \frac{\theta^2_1}{15}\right) \simeq \beta\theta^2\left[1 + \frac{1}{60}\left(4 - 9k\right)\theta^2\right]
  \label{eq:th1-th-approx}
\end{align}

La solución a primer orden (ignorando el término cuártico) es $\theta^2_ 1 \simeq \beta\theta^2$. Sustituímos esta solución en el
término correctivo y resolvemos para $\theta^2_1$:

\begin{align}
  \theta^2_1 &\simeq \beta\theta^2\left[1 + \frac{1}{60}\left(4 - 9k\right)\theta^2\right]\left(1 + \frac{\beta\theta^2}{15}\right)^{-1} \\
  \theta^2_1 &\simeq \beta\theta^2\left[1 + \frac{1}{60}\left(4 - 9k\right)\theta^2\right]\left(1 - \frac{\beta\theta^2}{15}\right) \\
  \implies \theta^2_1 &\simeq \beta\theta^2\left(1 + 2C_{k\beta}\theta^2\right) \\
  \mathrm{Donde:\quad}C_{k\beta} &= \frac{1}{2}\left(A_k - \frac{\beta}{15}\right) \\
  A_k &= \frac{1}{15} - \frac{3k}{20}
\end{align}

Utilizamos esta solución para $\theta_1$ en la ecuación (\ref{eq:R-geometric}), ignorando términos de orden superior al cuártico
dentro de los corchetes:

\begin{align}
  \theta_1 &\simeq \beta^{1/2}\theta\left(1 + 2C_{k\beta}\theta^2\right)^{1/2} \\
  \implies \theta + \theta_1 &\simeq \theta\left[1 + \beta^{1/2}\left(1 + 2C_{k\beta}\theta^2\right)^{1/2}\right]
  \simeq \theta\left[1 + \beta^{1/2}\left(1 + C_{k\beta}\theta^2\right)\right]\\ 
  \sin\theta_1 &\simeq \theta_1\left(1 - \frac{\theta^2_1}{6}\right) \\
  &\simeq \beta^{1/2}\theta\left(1 + 2C_{k\beta}\theta^2\right)^{1/2}\left[1 - \frac{\beta\theta^2\left(1 + 2C_{k\beta}\theta^2\right)}{6}\right]\\
  &\simeq \beta^{1/2}\theta\left(1 + C_{k\beta}\theta^2\right)\left[1 - \frac{\beta\theta^2\left(1 + 2C_{k\beta}\theta^2\right)}{6}\right]\\
           &\simeq \beta^{1/2}\theta\left[\left(1 + C_{k\beta}\theta^2\right) - \frac{\beta\theta^2\left(1 + 2C_{k\beta}\theta^2\right)
             \left(1 + C_{k\beta}\theta^2\right)}{6}\right]\\
  \implies \sin\theta_1 &\simeq \beta^{1/2}\theta\left[1 + \left(C_{k\beta}- \frac{\beta}{6}\right)\theta^2\right] \\
  \sin\left(\theta + \theta_1 \right) &\simeq \left(\theta + \theta_1\right)\left[1 - \frac{\left(\theta + \theta_1\right)^2}{6}\right] \\
           &\simeq \theta\left[1 + \beta^{1/2}\left(1 + C_{k\beta}\theta^2\right)\right] \left[1 - \frac{\theta^2\left(1 + \beta^{1/2}
             \left(1 + C_{k\beta}\theta^2\right)\right)^2}{6}\right] \\
           &\simeq \theta\left[1 + \beta^{1/2}\left(1+ C_{k\beta}\theta^2\right)\right]\left[1 - \frac{\theta^2}{6}\left(1 + \beta^{1/2}\right)^2
             \left(1 + \frac{2\beta}{1 + \beta^{1/2}}C_{k\beta}\theta^2\right)\right] \\
           &\simeq \theta\left[1 + \beta^{1/2} - \frac{1}{6}\left(1+\beta^{1/2}\right)^3\theta^2 + \beta^{1/2}C_{k\beta}\theta^2\right] \\
           &\simeq \theta\left(1 + \beta^{1/2}\right)\left[1 + \left(\frac{\beta^{1/2}C_{k\beta}}{1 + \beta^{1/2}} - \frac{1}{6}\left(1 + \beta^{1/2}\right)^2
             \theta^2\right)\right]\\
  \implies \frac{R}{D} &\simeq \frac{\beta^{1/2}}{1 + \beta^{1/2}}\left[1 + \left(C_{k\beta} - \frac{\beta}{6}\right)\theta^2\right]
                         \left[1 + \left(\frac{\beta^{1/2}C_{k\beta}}{1 + \beta^{1/2}} - \frac{\left(1 + \beta^{1/2}\right)^2}{6}\right)\right]^{-1}
\end{align}



\subsubsection{Interacción con un viento plano--paralelo}

\section{Proyección en el Plano del Cielo}
\label{sec:projection}

Para un choque de proa que es la vez geométricamente delgado y
ópticamente delgado, únicamente se observa el borde de éste por
abrillantamiento al limbo, por lo tanto, su orientación respecto a
la línea de visión modifica su forma respecto a la forma real del
choque. Para ello, rotamos el sistema de referencia del choque de proa
en coordenadas cartesianas, denotado por $(x, y, z)$, por un ángulo
que llamamos \textit{inclinación}, denotado por $i$, en el plano $xz$,
de modo que la transformación entre el sistema de refencia del choque
y el sistema de referencia del plano del cielo, denotado por
$(x', y', z')$ queda como sigue:

\begin{align}
  \left(
  \begin{array}{c}
    x' \\ y' \\ z'
  \end{array}
  \right) &=
  \left(
  \begin{array}{c}
    x\cos i - z\sin i \\ y' \\ z\cos i + x\sin i
  \end{array}
  \right)
  \label{eq:rotation}
\end{align}

Por otro lado, la forma tridimensional del choque de proa viene dado por:

\begin{align}
  \left(
  \begin{array}{c}
    x \\ y \\ z
  \end{array}
  \right) &=
            R(\theta)\left(
            \begin{array}{c}
              \cos\theta \\
              \sin\theta\cos\phi \\
              \sin\theta\sin\phi
            \end{array}
            \right)
\end{align}
La relación entre ambos sistemas de referencia se ilustra en la figura
\ref{fig:reference}.

\begin{figure}
  \includegraphics[width=\linewidth]{./Figures/projection-pos}
  \label{fig:reference}
  \caption{Sistema de referencia del choque vs sistema de referencia del
    plano del cielo. Los ejes $x'$ y $y'$ se encuentran en el plano del
    cielo, mientras el eje $z'$ es paralelo a la línea de visión.
    Solo la regi\'on del choque cuya tangente sea paralela a la l\'inea
    de visión será visible por abrillantamiento al limbo.}
\end{figure}

\subsection{Vectores normal y tangente a la superficie}

Si definimos los vectores $\hat{n}$ y $\hat{t}$, como los vectores
normal y tangente a la superficie, respectivamente para $\phi$ constante.
En el caso $\phi = 0$ (figura \ref{fig:unit-vec}), ambos vectores se encuentran
en el plano $xy$ y es fácil mostrar que:

\begin{align}
  \hat{t}_0 =
  \left(
  \begin{array}{c}
    -\cos\alpha \\
    \sin\alpha \\
    0
  \end{array}
  \right)
  \quad \mathrm{y} \quad
  \hat{n}_0 =
  \left(
  \begin{array}{c}
    \sin\alpha \\
    \cos\alpha \\
    0
  \end{array}
  \right)
  \label{eq:unit-vec}
\end{align}

Donde:
\begin{align}
  \tan\alpha = -\frac{dy}{dx} = \frac{1+\omega\tan\theta}{\tan\theta-\omega}
\end{align}
y:
\begin{align}
  \omega(\theta) = -\frac{1}{R}\frac{dR}{d\theta} 
\end{align}

Para otros valores de $\phi$, basta con hacer una rotación de las ecuaciones
(\ref{eq:unit-vec}) alrededor del eje $x$. Para la conversión al sistema de
referencia del plano del cielo se utiliza la ecuación (\ref{eq:rotation}):

\begin{align}
  \hat{n}' &= \frac{1}{\left(1 + \omega^2\right)^{1/2}} \\
           & \times \left(
             \begin{array}{c}
               (\cos\theta+\omega\sin\theta)\cos i-(\sin\theta-\omega\cos\theta)\sin i\sin\phi\\
               (\sin\theta-\omega\cos\theta)\cos\phi \\
               (\cos\theta+\omega\sin\theta)\sin i+(\sin\theta-\omega\cos\theta)\sin\phi\cos i
             \end{array}
                    \right) \\
    \hat{t}' &= \frac{1}{\left(1 + \omega^2\right)^{1/2}} \\
           & \times \left(
             \begin{array}{c}
               -(\sin\theta-\omega\cos\theta)\cos i-(\cos\theta+\omega\sin\theta)\sin i\sin\phi\\
               (\cos\theta+\omega\sin\theta)\cos\phi \\
               -(\cos\theta+\omega\sin\theta)\sin i+(\sin\theta-\omega\cos\theta)\sin\phi\cos i
             \end{array}
             \right) 
\end{align}


\begin{figure}
  \includegraphics[width=0.8\linewidth]{./Figures/bowshock-unit-vectors}
  \label{fig:unit-vec}
  \caption{Vectores unitarios normal y tangente a la superficie $R(\theta)$
    en un plano de azimuth $\phi$ constante.}
\end{figure}


\subsection{Línea tangente}

Debido a que el choque es ópticamente delgado y geométricamente
delgado, solo la región del choque cuya tangente sea paralela a la
línea de visión seré visible. Esto corresponde a una curva que
denominamos \textit{línea tangente}, que debe cumplir con la siguiente
condición:

\begin{align}
  \hat{n}'\boldsymbol{\cdot} \hat{z}' = 0
\end{align}

Denotamos como $\phi_T$ al ángulo azimutal que cumple la condición anterior
para una inclinación dada, en función del ángulo polar $\theta$:
\begin{align}
  \sin\phi_T = \tan i\tan\alpha = \tan i\frac{1+\omega\tan\theta}{\omega-\tan\theta}
  \label{eq:phi-tan}
\end{align}
De esta manera, la forma de la línea tangente del choque de proa, a la que llamamos
\textit{forma proyectada} viene dada por:

\begin{align}
  \left(
  \begin{array}{c}
    x'_T \\
    y'_T \\
    z'_T
  \end{array}
  \right) =
  R(\theta)\left(
  \begin{array}{c}
    \cos\theta\cos i - \sin\theta\sin\phi_t\sin i \\
    \sin\theta\left(1-\sin^2\phi_T\right)^{1/2} \\
    \cos\theta\sin i + \sin\theta\sin\phi_T\cos i
  \end{array}
  \right) \label{eq:proj-shape}
\end{align}
En el caso general, $z'_T$ no es una función lineal de $x'_t$ y $y'_T$, por lo que
la línea tangente no se encuentra en un plano.

La forma aparente $(x'_t, y'_T)$  de la línea tangente también puede escribirse
en coordenadas polares $(R', \theta')$, donde:
\begin{align}
  R'(\theta) = \left(x'_t^2 + y'_T^2\right)^{1/2} & \mathrm{y} & \tan\theta' = \frac{y'_T}{x'_T}
  \label{eq:polar}
\end{align}
Es de notar a su vez que la ecuación (\ref{eq:phi-tan}) no tiene solución para valores
arbitrarios de $\theta$ y de la inclinación, puesto que se requiere que
$\left|\sin\phi_T\right| < 1$. Por tanto, la línea tangente solo existe para valores
de $\theta$ tales que $\theta < \theta_0$ donde $\theta_0$ es el valor de $\theta$ en
el eje de simetría de la línea tangente proyectada $(\theta'(\theta_0)) = 0$ y que se
obtiene de la siguiente ecuación implícita:
\begin{align}
  \tan\theta_0 = \frac{|\tan i| + \omega(\theta_0)}{1 - \omega(\theta_0)|\tan i|}
  \label{eq:th-0}
\end{align}
Esto implica que si el choque de proa es suficientemente ``abierto''
$(\alpha > \alpha_{min})$, entonces para inclinaciones tales que
$|i| > 90^\circ - \alpha_{min}$ no existirá la línea tangente para ningún valor de $\theta$,
es decir, el choque de proa se encontrará sufientemente ``de cara'' como para que ya no
parezca un chouque de proa para el observador.

\subsection{Radios característicos en el plano del cielo}

En orden de comparar la forma $R(\theta)$ con observaciones, es útil definir los radios
característicos $R'_0$ y $R'_{90}$, donde $R'_0$ es el radio del eje de simetría aparente
y $R'_{90}$ es el radio aparente en la dirección perpendicular a $R'_0$. Es decir
$R'_0 = x'_T(y'_t=0)$ y $R'_{90} = y'_t(x'_t = 0)$. Utilizando las ecuaciones
(\ref{eq:phi-tan}) y (\ref{eq:proj-shape}) encontramos que:
\begin{align}
R'_0 = R(\theta_0)\cos(\theta_0 + i)
\label{eq:R0p}
\end{align}
Donde $\theta_0$ es la solución de la ecuación (\ref{eq:th-0}), y
\begin{align}
  R'_{90} = R(\theta_{90})\sin\theta_{90}\left(1-\sin^2\phi_T(\theta_{90})\right)^{1/2}
  \label{eq:R90p}
\end{align}
donde $\theta_{90}$ es la solución de la siguiente ecuación implícita:
\begin{align}
  \cot\theta_{90} = \frac{1 - \left(1+\omega(\theta_{90})^2\sin^22i\right)^{1/2}}
  {2\omega(\theta_{90})\cos^2i}
  \label{eq:th90}
\end{align}

\section{Cuádricas de Revolución}

\newcommand\Sin{\ensuremath{\mathcal{S}}}
\newcommand\Cos{\ensuremath{\mathcal{C}}}
\newcommand\Cot{\ensuremath{\mathcal{T}}}


En el caso general es difícil encontrar la forma aparente para un choque de
proa siguiendo el formalismo desarrollado en la sección anterior, por lo que
optamos por aproximar la forma éstos con una de las superficies más simples:
las \textit{cuádricas de revolución}, que son superficies de revolución de
las curvas cónicas. Dado el modelo general descrito en la sección
\ref{sec:Modelo-generico}, haremos algunas restricciones para las superficies
cuádricas que utilizaremos en este trabajo:
\begin{itemize}
  \item El eje focal se encuentra alineado con el eje $x$
  \item La posición del foco de la superficie cuádrica no necesariamente coincide
    con la posición de la fuente
  \item En el caso de las hipérbolas, solo tomamos una de las ramas de ésta.
\end{itemize}
Implementando dichas restricciones, utilizamos la representación paramétrica de
las curvas cónicas en términos de un parámetro adimensional denotado con la letra
$t$ de manera general:
\begin{align}
  x &= a\Cos(t) + x_0\\
  y &= b\Sin(t) 
\end{align}
Donde:
\begin{align}
  \Cos(t) =\left\lbrace
  \begin{array}{lr}
    \cos{t} & \theta_c > 0\\
    -\cosh{t} & \theta_c < 0        
  \end{array}\right. \\
  \Sin(t) = \left\lbrace
  \begin{array}{lr}
    \sin{t} & \theta_c > 0\\
    \sinh{t}  & \theta_c < 0
  \end{array} \right. \\
  x_0 &= R_0 \mp a \label{eq:x0} 
\end{align}
Donde $a$ y $b$ representan la longitud del semi-eje mayor y menor, respectivamente (Figura \ref{fig:conics}).
$x_0$ representa la distancia entre el centro de la cónica y el origen. 
$\theta_c$ es un parámetro que está relacionado con la excentricidad y que en este
trabajo sustituye a ésta y que están relacionadas por la siguiente expresión:
\begin{align}
  \tan\theta_c = \pm\sqrt{\left|1-e^2\right|}
\end{align}
Tomamos el signo positivo cuando la cantidad dentro de las barras de valor absoluto es
positiva y viceversa (Figura \ref{fig:conics-family}). También podemos definirlo en
términos de los parámetros de las cónicas:

\begin{align}
  \tan\theta_c = \pm \frac{b}{a} \label{eq:thc}
\end{align}
Siguiendo la convención de que el signo positivo corresponde a elipses, negativo a
hipérbolas y cero para las parábolas.

\begin{figure}
  \begin{tabular}{cc}
    \includegraphics[width=0.5\linewidth]{./Figures/ellipse_edited} &
    \includegraphics[width=0.5\linewidth]{./Figures/hyperbola_edited}
  \end{tabular}
  \label{fig:conics}
  \caption{Representación esquemática de: Izquierda: Elipse. Y, derecha: Hipérbola.
  En ambos casos se ilustran los parámetros relevantes de éstas y los radios característicos}
\end{figure}

\begin{figure}
  \includegraphics[width = \linewidth]{./Figures/conic1}
  \label{fig:conics-family}
  \caption{Familia de curvas cónicas, donde el valor del parámetro $\theta_c$ varía desde
    $\theta_c <0$ (hipérbolas) hasta $\theta_c > 0$ (elipses). Casos especiales son $\theta_c = 0$
  (parábola) y $\theta_c = 45^\circ$ (círculo). Este parámetro sustituye en este trabajo a la excentricidad.}
\end{figure}

\subsection{Radios Característicos}

Para que las curvas cónicas den una buena aproximación a la forma de un choque de proa dado, necesitamos saber
calcular los radios característicos para éstas. A partir de la descripción de estos en la sección
\ref{sec:char-rad} podemos encontrar expresiones para cada uno de éstos en términos de los parámetros de las
cónicas:

\begin{align}
  R_c &= \frac{b^2}{a} \label{eq:R-curv-conic}\\
  R_{90} &= b\left[\pm\left(1 - \frac{(R_0 - a)^2}{a^2}\right)\right]^{1/2} \label{eq:R90-conic}
\end{align}
$R_0$ es independiente de los parámetros de las cónicas, por tanto, en
esta sección nos será útil normalizar con este radio. De esta forma,
podemos invertir las siguientes ecuaciones:

\begin{align}
  \tilde{a} &= \pm\frac{\tilde{R}_c}{2\tilde{R}_c - \tilde{R}_{90}^2} \\
  \tilde{b} &= \frac{\tilde{R}_c}{\left|2\tilde{R}_c - \tilde{R}_{90}^2\right|^{1/2}}\\
  \tan\theta_c &= \pm\left|2\tilde{R}_c - \tilde{R}_{90}^2\right|^{1/2}
\end{align}
Nótese que la cantidad $T_c\equiv 2\tilde{R}_c - \tilde{R}_{90}^2$ nos sirve como discriminante para
distinguir el tipo de curva cónica que mejor ajusta a un choque de proa dado.

\subsection{Proyección en el plano del cielo}

El objetivo de esta sección es obtener la forma proyectada de las cuádricas de revolución,
puesto que son una aproximación buena y mucho más sencilla a la forma real de un choque de
proa.
La forma tridimensional de las cuádricas de revolución viene dada por:

\begin{align}
  x &= a\Cos(t) - x_0 \\
  y &= b\Sin(t)\cos\phi \\
  z &= b\Sin(t)\sin\phi
\end{align}

Siguiendo el procedimiento mostrado en la sección \ref{sec:projection} calculamos el ángulo
azimutal $\phi$ que cumple con el criterio de ser tangente al la línea de visión:

\begin{align}
  \sin\phi_T = \frac{b}{a}\tan i\Cot(t) 
\end{align}
Donde:
\begin{align}
  \Cot(t) = \left\lbrace
  \begin{array}{lr}
    \cot t & if~\theta_c > 0 \\
    \coth t & if~\theta_c < 0 
  \end{array}
  \right.
\end{align}

Podemos movernos a otro sistema de referencia $(X, Y)$ centrado en el origen, donde
$X = x - x_0$ y $Y =y$. En este sistema, utilizamos la ecuación (\ref{eq:rotation})
para obtener la forma aperente de una cuádrica dada:

\begin{align}
  X'_T &= \frac{\Cos(t)}{a\cos i}\left(a^2\cos^2 i \pm b^2\sin^2 i\right)
  \label{eq:x-prime-proj}\\
  Y'_T &= b\Sin(t)\left(1 - \frac{b^2}{a^2}\tan^2 i\Cot^2(t)\right)^{1/2}
  \label{eq:y-prime-proj}
\end{align}

Se espera que la forma proyectada de una cuádrica dada sea otra cuárica del mismo
tipo, por lo que es posible escribir las ecuaciones (\ref{eq:x-prime-proj}) y
(\ref{eq:y-prime-proj}) de la siguiente manera: 

\begin{align}
  X'_T &= a'\Cos(t') \\
  Y'_T &= b'\Sin(t')
\end{align}
Donde:
\begin{align}
  a' &= \left(a^2\cos^2 i \pm \b^2\sin^2 i\right)^{1/2} \label{eq:a-prime}\\
  b' &= b \label{eq:b-prime}\\
  \Cos(t') &= \frac{a'\Cos(t)}{a\cos i} \\
  \Sin(t') &= \left(1 - \Cos^2(t')\right)^{1/2}
\end{align}

También implica que para encontrar los radios característicos en el sistema de
referencia del observador solamente tenemos que sustituir $a$ y $b$ por $a'$ y
$b'$ en las ecuaciones (\ref{eq:x0}), (\ref{eq:R-curv-conic}), (\ref{eq:R90-conic})
y (\ref{eq:thc}):

\begin{align}
  R'_0 &= \pm a' +x_0\cos i\\
  R'_c &= \frac{b'^2}{a'}\\
  \tan\theta'_c &= \frac{b'}{a'} \\
  R'_{90} &= \left(2R'_c \mp \tan^2\theta'_c\right)^{1/2}
\end{align}

Utilizando las ecuaciones (\ref{eq:x0}), (\ref{eq:a-prime}) y (\ref{eq:b-prime}),
utlizando la definición $D' = D\cos i$ e introduciendo la función
$f(i;\theta_c)\equiv \left(1 \pm \tan^2\theta_c\tan^2i\right)^{1/2}$ obtenemos
ecuaciones explícitas para los radios característicos en el sistema de referencia
del plano del cielo en términos de la inclinación:

\begin{align}
  \frac{q'}{q} &= 1 \pm \tilde{R}_c\cot^2\theta_c\left(f(i;\theta_c) - 1\right) \\
  \tilde{R}'_c &= \frac{\tilde{R_c}}{\cos^2if(i;\theta_c)\frac{q'}{q}} \\
  \tan\theta'_c &= \frac{\tan\theta_c}{\cos if(i;\theta_c)} \\
  \tilde{R}'_{90} &= \left(\frac{2\tilde{R}_cf(i;\theta_c) \mp
                  \tan^2\theta_c\frac{q'}{q}}{q'/q}\right)^{1/2}\frac{\sec i}{f(i;\theta_c)}
\end{align}

Cuando $\tilde{R}'_{90}$ es medible, entonces es posible hacer diagramas de diagnóstico como
el de la figura \ref{fig:diagnostic} para comparar con observaciones, independientemente de
cualquier modelo de choques de proa.

\begin{figure}
  \includegraphics[width=0.5\linewidth]{./Figures/projected-R90-vs-Rc}
  \label{fig:diagnostic}
  \caption{Diagrama de diagnóstico $\tilde{R}'_{90}$ vs $\tilde{R}'_c$ para las cuádricas
    de revolución. En la región sin sombrear se representan las superficies abiertas
    (hiperboloides, $\theta_c <0$), mientras que la región más oscura representa  a
    elipsoides prolatos  $(0 < \theta_c < 45^\circ)$ y la región poco sombreada a
    elipsoides oblatos $(\theta_c > 45^\circ)$}
\end{figure}

Buscamos adjuntar el paper ``quadrics bowshock''
