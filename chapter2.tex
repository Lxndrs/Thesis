\section{Vientos Estelares}
\section{Choques}
\section{Frentes de Ionizaci\'on}
\section{Regiones HII}
\section{Aproximación Hipers\'onica}
\label{sec:hipersonica}
\section{Modelo Genérico de los Choques de Proa}
\label{sec:Modelo-generico}
Para este trabajo consideramos en general dos modelos de
interacci\'on  de vientos:
\begin{itemize}
\item Una fuente localizada en el origen que emite un viento esf\'erico
  que puede ser isotr\'opico o anisotr\'opico (figura
  \ref{fig:isotropic-aniso}) no acelerado que interact\'ua con el viento
  esf\'erico isotr\'opico de otra fuente que se encuentra a una distancia
  $D$ de la primera(figura \ref{fig:crw-esquema})
\item Una fuente localizada en el origen que emite un viento esf\'erico
  isotr\'opico no acelerado que interact\'ua con un viento plano paralelo
  no acelerado y densidad constante (figura )
\end{itemize}
El sitema en su conjunto tiene simter\'ia cil\'indrica.
\begin{figure}
  \includegraphics[width=0.5\linewidth]{./Figures/anisotropic-arrows}
  \label{fig:isotropic-aniso}
  \caption{Representaci\'on esquem\'atica de vientos con diferentes
    anisotrop\'ias:
    Arriba izquierda: Viento isotr\'opico esf\érico. Arriba derecha: viento
    isotr\'opico hemisf\'erico. Abajo: Vientos anisotr\'opicos donde el
    par\'ametro $k$ indica el grado de anisotrop\'ia (ver secci\'on
    \ref{sec:hipersonica})}
\end{figure}
\begin{figure}
  \includegraphics[width=0.5\linewidth]{./Figures/bowshock-crw-variables}
  \label{fig:crw-esquema}
\end{figure}

\subsection{Radios ``Caracter\'isticos''}

Las cantidades medibles que nos ayudan a caracterizar un choque de proa las
llamamos ``Radios caracter\'isticos'' (ilustrados en la figura
\ref{fig:char-radii}):
\begin{itemize}
\item Radio del choque en la direcci\'on del eje de simetr\'ia del sistema.
  Denotado como $R_0$
\item Radio en direcci\'on perpendicular al eje de simetr\'ia del sistema.
  Denotado como $R_{90}$
\item Radio de curvatura en la ``nariz'' del choque de proa. Denotado
  como$ R_c$
\end{itemize}

\begin{figure}
  \includegraphics[width=\linewidth]{./Figures/characteristic-radii}
  \label{fig:char-radii}
  \caption{Representaci\'on esquem\'atica de los radios caracter\'isticos
  de un choque de proa}
\end{figure}



\section{Proyecci\'on en el Plano del Cielo}

Para un choque de proa que es la vez geom\'etricamente delgado y
\'opticamente delgado, \'unicamente se observa el borde de \'este por
abrillantamiento al limbo, por lo tanto, sua orientaci\'on respecto a
la l\'inea de visi\'on modifica su forma respecto a la forma real del
choque. Para ello, rotamos el sistema de referencia del choque de proa
en coordenadas cartesianas, denotado por $(x, y, z)$, por un \'angulo
que llamamos \textit{inclinaci\'on}, denotado por $i$, en el plano $xz$,
de modo que la transformaci\'on entre el sistema de refencia del choque
y el sistema de referencia del plano del cielo, denotado por
$(x', y', z')$ queda como sigue:

\begin{align}
  \left(
  \begin{array}
    x' \\ y' \\ z'
  \end{array}
  \right) &=
  \left(
  \begin{array}
    x\cos i - z\sin i \\ y' \\ z\cos i + x\sin i
  \end{array}
  \right)          
\end{align}


\section{Cu\'adricas de Revoluci\'on}

Buscamos adjuntar el paper ``quadrics bowshock''
