\chapter[Resultados Obtenidos]{Resultados obtenidos para los proplyds ``clásicos''}
\label{chap:proplyds}
\thispagestyle{empty}
Probamos nuestro modelo descrito en los capítulos anteriores en una muestra de proplyds pertenecientes a la Nebulosa de Orión (ONC) que presentan un choque de proa. En la figura \ref{fig:proplyds-map} se muestran los proplyds que pertencen a nuestra muestra.

En todos los casos no fue posible medir el radio característico $R_{90}$ debido a que el brillo de la cáscara decae con el ángulo polar $\theta$ y no es detectable para ángulos del orden de $60^\circ$. Sin embargo, a continuación mostraremos la metodología para obtener la inclinación más probable de cada choque, así como los parámetros del modelo de cada uno de éstos que nos indican su forma intrínseca. Los resultados mostrados en este capítulo forman parte de un artículo por publicar.

\begin{figure*}
  \centering
    \includegraphics[width=\linewidth]{./Figures/LV-full-field-annotated}
    \caption{Imagen de la parte central de la Nebulosa de Orión donde se ubican los proplyds de nuestra muestra. Las cruces color cyan corresponden a las mediciones de la forma aparente para cada choque de proa. Los círculos amarillos marcan la posición de cada proplyd y la ``x'' roja corresponde a la posición de la estrella ionizante \thC{}. Los círculos negros ilustran de manera esquemática el radio de curvatura de cada choque.}
    \label{fig:proplyds-map}
\end{figure*}

\section{Metodología para la medición de la forma aparente.}
\label{sec:methodology}
Se utilizaron imágenes en el filtro de [\Ion{O}{III}] de la cámara WPC2 del Telescopio Espacial Hubble (HST). Se utilizaron las herramientas del programa DS9 para análisis de imágenes astronómicas para trazar la posición de \thC{} y de cada uno de los proplyds de la muestra. La posición y la forma de los choques de proa fue trazada con una serie de marcas a lo largo del choque. Las coordenadas de las marcas fueron guardadas en un archivo y luego procesadas para tener las coordenadas del choque en el sistema de referencia del proplyd (Figura \ref{fig:proplyds-map}). El radio de curvatura aparente se obtiene haciendo un ajuste de mínimos cuadrados de la forma de un círculo de las mediciones obtenidas. $R_0$ se obtiene como la distancia a lo largo del eje $x$ entre el proplyd y el ajuste circular dentro del rango de las coordenadas de las mediciones. 

\subsection{Medición de incertidumbres}

Para saber qué tan confiables son las coordenadas de las mediciones, se realizó el procedimiento siguiente: Del total de mediciones realizadas para cada proplyd, se crearon varias sub-muestras donde se utilizamos aproximadamente las dos terceras partes de las mediciones, pero dejando un mínimo de cuatro puntos, y se procedió a calcular los radios característicos para cada sub-muestra, y comprobar qué tanto se desvían estas mediciones de la original. En la figura \ref{fig:char-radii-obs} se muestran ejemplos de dichas sub-muestras para algunos proplyds.


\begin{figure*}
  \centering
  \setkeys{Gin}{width=0.33\linewidth}%, trim=10 30 55 62.5}
\begin{tabular}{@{}c@{}c@{}c@{}}
 
Todos los puntos & Primera sub-muestra & Segunda sub-muestra \\ \includegraphics[clip]{./Programs/LV-bowshocks-xyfancy-positionswill-177-341} & \includegraphics[clip]{./Programs/Multi-Fit/samp00/LV-bowshocks-xyfancy-positionssamp00-177-341} &
\includegraphics[clip]{./Programs/Multi-Fit/samp01/LV-bowshocks-xyfancy-positionssamp01-177-341} \\
\includegraphics[clip]{./Programs/LV-bowshocks-xyfancy-positionswill-LV4} & \includegraphics[clip]{./Programs/Multi-Fit/samp00/LV-bowshocks-xyfancy-positionssamp00-LV4} & \includegraphics[clip]{./Programs/Multi-Fit/samp01/LV-bowshocks-xyfancy-positionssamp01-LV4} \\
\includegraphics[clip]{./Programs/LV-bowshocks-xyfancy-positionswill-168-328} &  \includegraphics[clip]{./Programs/Multi-Fit/samp00/LV-bowshocks-xyfancy-positionssamp00-168-328} & \includegraphics[clip]{./Programs/Multi-Fit/samp01/LV-bowshocks-xyfancy-positionssamp01-168-328}
\end{tabular}
\caption{Ejemplos de incertidumbres sistemáticas en los ajustes circulares a la forma de los choques para tres fuentes (desde la línea superior hasta la inferior): 177-341, LV4 y 168-328. La columna de la izquierda muestra el ajuste a todos los puntos identificados en el borde de la cáscara, donde el número y el espaciamiento de los puntos es una medida subjetiva de nuestra confianza al trazar el borde de cada cáscara. Las dos columnas restantes muestran ajustes a sub-muestras seleccionadas aleatoriamente que contienen 2/3 partes de los puntos de la muestra original para cada cáscara.}
\label{fig:char-radii-obs}
\end{figure*}

\section{Resultados Empíricos.}

Los radios característicos obtenidos para la muestra original y para las submuestras se muestran en la figura \ref{fig:obs-diagnostic}. En cada pánel se utiliza un valor fijo para el parámetro de anisotropía $k$. De estas figuras se pueden obtener la tasa de momentos $\beta$ y la inclinación $i$ para un grado de anisotropía dado por inspección visual al encontrar intersecciones entre las curvas teóricas y las barras radiales de las incertidumbres de cada proplyd. En general algunas observaciones cualitativas que se encuentran son: Los proplyds con planitud mayor, LV4 y LV2b ajustan mejor a modelos donde el parámetro de anisotropía es bajo. LV2, quien tiene la planitud aparente más baja de toda la muestra, ajusta con modelos con alto índice de anisotropía $(k \gtrsim 3)$, con HST1 (177-341) ocurre algo similar, sin embargo, los modelos a los que ajusta este proplyd tienen una tasa de momentos baja e inclinaciones muy altas $(i \sim 80^{\circ})$. Esto es difícil de atribuirselo a errores en las mediciones por que es el proplyd que menos desviaciones tiene entre la medición original y las sub-muestras. El resto de los proplyds ajusta bien con un parámetro de anisotropía medio $(k\sim 1/2 - 3)$. Dependiendo de los parámetros $(\beta, k)$, la inclinación que se le puede atribuir a cada proplyd en la mayoría de los casos varía entre $15^\circ$ y $40^\circ$. 

\begin{figure}
  \centering
 \includegraphics[width=0.48\linewidth]{./Figures/obs-diagnostic-Pi-R0-Cantoid} & \includegraphics[width=0.48\linewidth]{./Figures/obs-diagnostic-Pi-R0-k05} \\
  \includegraphics[width=0.48\linewidth]{./Figures/obs-diagnostic-Pi-R0-k30} & \includegraphics[width=0.48\linewidth]{./Figures/obs-diagnostic-Pi-R0-k80}
  \caption{Similar a la figura \ref{fig:Lambda-Pi-diagram} pero sustituyendo la alatud aparente por el radio aparente en el ápex $R'_0/D'$ para diferentes grados de anisotropía $k$, donde en cada pánel se asume que este parámetro es fijo. A lo largo de cada curva el valor del parámetro $\beta$ es fijo,  mientras que la inclinación se incrementa a lo largo de la curva, empezando a partir del círculo grande, donde $i=0^\circ$. Las marcas circulares pequeñas representan intervalos de $15^\circ$, mientras que las marcas más pequeñas representan intervalos de $5^\circ$. Los resultados observacionales de los choques de proa para nuestro set de proplyds se muestran con puntos negros, mientras que las mediciones de las sub muestras se muestran con líneas de colores radiales que parten desde la medición ``principal''. La opacidad de la medición de cada sub muestra es mayor cuanto menor sea la desviación respecto a la medición principal.}
  \label{fig:obs-diagnostic}
\end{figure}
%\begin{figure*}
%\begin{tabular}{cc}
%\includegraphics[width=0.48\linewidth]{./Figures/conic_xi-10} & \includegraphics[width=0.48\linewidth]{./Figures/conic_xi-08} \\
%\includegraphics[width=0.48\linewidth]{./Figures/conic_xi-04} & \includegraphics[width=0.48\linewidth]{./Figures/conic_xi-02} 
%\end{tabular}
%\caption{Mediciones de los radios característicos de los proplyds $R_c$ y $R_0$. Las curvas representan el ajuste de una cuádrica para un choque de proa con un cociente de momentos $\beta$ fijo, además se muestra su respectivo valor de $\theta_c$. Los puntos a lo largo de cada curva representan una separación en inclinación de $15^\circ$. Las mediciones para cada proplyd vienen acompañadas con el set de sub-muestras representadas como líneas radiales de colores. En cada gráfica se utiliza un valor diferente para el parámetro de anisotropía $\xi$, iniciando con un viento isotrópico $(\xi=1)$, hasta el viento  con mayor anisotropía $(\xi=0.2)$.}
%\label{fig:conic-xi}
%\end{figure*}

Con base a este análisis, se resume en la tabla \ref{tab:arc-fits} los ajustes a los parámetros de los proplyds: cociente de momentos $\beta$, inclinación, distancia a \thC{} intrínseca $D$ y radio del choque en el ápex $R_0/D$.
\begin{landscape}
  \begin{table*}
    \centering
  \caption{Ajuste a los parámetros de los arcos para los choques de proa de los proplyds}
  \label{tab:arc-fits} 
  \newcommand\C[1]{\multicolumn{1}{c}{#1}}
  \begin{adjustbox}{width=1.35\textwidth}
    \small
\begin{tabular}{llrllllrlll}\toprule
             &          & \multicolumn{3}{c}{\dotfill Observado \dotfill}              & \multicolumn{6}{c}{\dotfill Ajuste teórico \dotfill} \\ 
  \C{OW}     & \C{Nombre} & \(D'\) &\C{ \(R_0'/D'\) }&\C{ \(\Pi'_{\mathrm{shape}}\) }&\C{ \(\Pi'_{\mathrm{flux}}\) }&\C{ \(\beta\) }&\C{ \(k\) }&\C{ \(|i|\) }&\C{ \(D\) }&\C{ \(R_0/D\)}\\
  \C{(1)}& \C{ (2) }&\C{ (3)    }&\C{    (4)      }&\C{              (5)           }&\C{           (6)             }&\C{     (7)   }&\C{   (8)   }&\C{   (9) }&\C{  (10) }&\C{   (11)} \\
\midrule     
 168-328  &            &    6.8  &  $0.15 \pm 0.01$  &  $1.45^{+0.10}_ {-0.15}$   &  $1.55 \pm 0.05$     &  0.005  &  0.5  &  $52.5 \pm 2.50$   &  $0.022 \pm \SI{1.5e-3}{}$  &  $0.07$  \\
 169-338  &            &  16.4  &  $0.06 \pm 0.01$  &  $1.45^{+1.05}_{-0.25}$   &  $1.65 \pm 0.10$     &  0.002  &  0.0 -- 0.5  &  $36.3 \pm 1.25$   &  $0.040 \pm \SI{1.3e-3}{}$  &  $0.04$  \\
 177-341  & HST1   & 25.6  &  $0.15 \pm 0.01$  &  $1.25 \pm 0.05$   &  $1.15 \pm 0.05$     &  0.0005 -- 0.001  &  3.0 -- 8.0  &  $72.5 \pm 2.50$   &  $0.171 \pm \SI{2.6e-2}{}$  &  $0.04$  \\
 180-331  &             &  25.1  &  $0.07^{+0.01}_{-0.03}$  &  $1.30 \pm 0.10$   &  $1.30 \pm 0.10$     &  0.0005  &  0.5  &  $62.5 \pm 2.50$   &  $0.109 \pm \SI{2.2e-3}{}$  &  $0.02$  \\
 167-317  &  LV2     &    7.8  &  $0.29^{+0.03}_{-0.05}$  &  $1.15^{+0.35}_{-0.55}$   &  $1.03 \pm 0.18$      &  0.02 -- 0.1  &  3.0 -- 8.0  &  $42.5 \pm 2.04$  &  $0.021 \pm \SI{9.2e-4}{}$  &  $0.18 \pm 0.06$  \\
 166-316  & LV2b    &   7.2  &  $0.11^{+0.01}_{-0.03}$  &  $1.75^{+0.85}_{-0.35}$   &  $1.75 \pm 0.10$     &  0.02 -- 0.01  &  0.0 -- 0.5  &  $20.0 \pm 2.50$  &  $0.015 \pm \SI{4.4e-4}{}$  &  $0.11 \pm 0.02$  \\
  163-317  & LV3      &   6.9  &  $0.33 \pm 0.01$  &  $1.80^{+0.30}_{0.10}$   &  $2.05 \pm 0.05$     &  0.06  &  0.5  &  $40.0 \pm 2.50$   &  $0.018 \pm \SI{9.0e-4}{}$  &  $0.20$  \\
 161-324  & LV4      &   6.2  &  $0.19 \pm 0.01$  &  $2.65^{+0.25}_{-0.65}$   &  $2.10 \pm 0.05$     &  0.02 -- 0.05  &  0.0  &  $23.8 \pm 13.75$  &  $0.014 \pm \SI{1.7e-3}{}$  &  $0.15 \pm 0.03$  \\
 158-323  & LV5      &   9.6  &  $0.21 \pm 0.01$  &  $1.55 \pm 0.15$   &  $1.70 \pm 0.05$     &  0.02  &  0.5  &  $42.5 \pm 2.50$   &  $0.026 \pm \SI{9.4e-3}{}$  &  $0.02$  \\
\bottomrule
\end{tabular}
\end{adjustbox}
\begin{minipage}{0.95\linewidth}
  \centering
\footnotesize
  Notas --
%
  Col.~(1): ID de la fuente \citep{ODell:1994a}.
%
  Col.~(2): Nombre alternativo de la fuente.
% 
  Col.~(3): Distancia proyectada desde \thC{}, segundos de arco.
%
  Col.~(4): Radio exterior aparente a lo largo del eje, normalizado con la distancia proyectada, donde la incertidumbre es calculada a partir de los valores máximo y mínimo de las submuestras descritas en \S~\ref{sec:methodology}, pero utlizando como mínimo la mitad de la resolución de los ejes de la figura \ref{fig:obs-diagnostic}. Se determina con el ajuste circular decrito en \S~\ref{sec:methodology}.
% 
  Col.~(5): Planitud aparente, donde la incertidumbre es calculada del mismo modo que en Col.~(4). Se determina con el ajuste circular descrito en \S~\ref{sec:methodology}.
% 
  Col.~(6): Planitud aparente, pero aplicando el criterio adicional de que el brillo superficial del proplyd obtenido debe coincidir con la predicción teórica. La medición central corresponde al promedio de las mediciones de las submuestras que cumplen con dicho criterio, con una desviación de $\pm 1\sigma$. Si solo una submuestra cumple el criterio, el resultado de Col.~(5) se traspasa a esta columna. 
%
  Col.~(7): Cociente de momentos entre el viento del proplyd y la estrella O (ver capítulo \ref{chap:hipersonica}) de las submuestras utilizadas en Col.~(6). 
% 
  Col.~(8): Parámetro de anisotropía del viento del proplyd.
% 
  Col.~(9): Inclinación respecto al plano del cielo, en grados.
% 
  Col.~(10): Distancia real desde \thC{}, parsecs.
%
  Col.~(11): Radio real de la cáscara a lo largo del eje, normalizado con distancia.

\end{minipage}
\end{table*}
\end{landscape}

\section{Obtención de la Presión de Equilibrio}

Las mediciones de los radios característicos y las inclinaciones obtenidas para los proplyds en esta sección se puede predecir la distancia intrínseca del proplyd $D$ a \thC{}, así como la escala intrínseca del proplyd, dada por el radio en el ápex $R_0$ (columnas 10 y 11 de la tabla \ref{tab:arc-fits}). Con ayuda de estos parámetros y además conociendo el radio del Frente de Ionización y la densidad máxima del flujo fotoevaporado en el Frente de Ionización de cada proplyd podemos estimar el flujo $\Nio_*$ que se requiere para que exista equilibrio de ionización y compararlo con el flujo ionizante que recibe el proplyd de \thC{} a la distancia $D$. A su vez se puede estimar la presión RAM del flujo fotoevaporado antes del choque y compararlo con la presión RAM del viento estelar de \thC{} antes del choque. 


%Utilizamos los perfiles de brillo obtenidos en \citet{HA:1998} para obtener la densidad máxima y el radio del Frente de Ionización de nuestra muestra de proplyds. Sin embargo, a partir de los datos mostrados en la tabla 2 de dicho artículo, encuentro que utilizan una distancia a \thC{} de \SI{460}{pc}, mientras que en este trabajo utilizo una distancia de $414 \pm 6.8 \mathrm{~pc}$ \citep{Menten:2007}. En la tabla \ref{tab:prop-IF-par} muestro la densidad máxima y el radio del Frente de Ionización de nuestra muestra de proplyds corregidos por distancia a \thC{}, además, \citet{Henney:2001} encuentra que la densidad máxima obtenida en \citet{HA:1998} está sobreestimada por un factor del orden del 33\%.


\begin{table}
  % \begin{adjustbox}{width=\linewidth}
  \centering
  \caption{Parámetros del Frente de Ionización de los proplyds.}
  \label{tab:Prop-IF-par}
  \begin{tabular}{cccc} \toprule
    OW      & Nombre & $r_{\mathrm{IF}, 14}$*     & $N_6$* \\
    \midrule
    168-328 &        & $2.5 \pm 0.3$  & $4.22^{+2.71}_{-0.51}$ \\
    169-338 &        & $2.5 \pm 0.3$  & $1.48^{+1.13}_{-0.22}$ \\
    177-314 & HST1   & $18.4 \pm 1.7$ & $0.43^{+0.15}_{-0.07}$ \\ 
    180-331 &        & $11.0 \pm 1.3$ & $0.51^{+0.32}_{-0.11}$ \\
    167-317 & LV2    & $7.1 \pm 0.4$  & $2.67^{+1.24}_{-0.30}$ \\
    166-316 & LV2b   & $2.2 \pm 0.6$  & $4.35^{+3.03}_{-1.34}$ \\
    163-317 & LV3    & $4.5 \pm 0.6$  & $3.28^{+1.98}_{-0.88}$ \\
    161-324 & LV4    & $3.1 \pm 0.3$  & $4.35^{+2.49}_{-0.77}$ \\
    158-323 & LV5    & $5.7 \pm 0.6$  & $2.46^{+1.45}_{-0.60}$ \\
  \bottomrule
  \end{tabular}
  \begin{minipage}{0.9\linewidth}
    \centering
    \footnotesize
    Notas --
%
  Col.~(1): ID de la fuente \citep{ODell:1994a}.
%
  Col.~(2): Nombre alternativo de la fuente.
%
  Col.~(3): Radio del IF del proplyd, en unidades de \SI{e14}{cm}
%
  Col.~(4): Densidad máxima del IF, en unidades de \SI{e6}{cm^{-3}}
%
  * En la Col.~(3) la corrección por distancia se da como sigue: $r_{\mathrm{IF}, 14} = r_{\mathrm{IF}, 14}^{\mathrm{HA}}\frac{\SI{414}{pc}}{\SI{460}{pc}}$, mientras que en la Col.~(4) la corrección es $N_6 = N_6^{\mathrm{HA}}\left(\frac{\SI{414}{pc}}{\SI{460}{pc}}\right)^{-1/2}$, donde además la cantidad $N_6^{\mathrm{HA}}$ está reducida a solo las dos terceras parte de lo que se reporta en \citet{HA:1998}.
  \end{minipage}
\end{table}

El flujo de fotones ionizantes necesario para que exista equilibrio de ionización en el flujo fotoevaporado proveniente del proplyd, se puede determinar con la siguiente ecuación \citep{Henney:2001}:

\begin{align}
  F_{\mathrm{ph}} = v_w(r_{\mathrm{IF}}) n_{\mathrm{IF}} + \alpha'_{\mathrm{rec}}n^2_{\mathrm{IF}} \omega r_{\mathrm{IF}}\label{eq:F-ph}
\end{align}

Donde $\alpha'_{\mathrm{rec}} = \SI{2.6e-13}{cm^3.s^{-1}}$ (ver apéndice \ref{app:HII}), $\omega \simeq 0.12$ es un factor que está relacionado con la ley de velocidades del flujo fotoevaporado del proplyd, y es proporcional al grosor del IF escalado con $r_{\mathrm{IF}}$, $n_{\mathrm{IF}}$ es la densidad máxima del Frente de Ionización, y por último, estamos considerando que los Frentes de Ionización son de tipo D crítico, por lo que $v_w(r_{\mathrm{IF}}) = a_\marhrm{II} = \SI{11}{km.s^{-1}}$ es la velocidad del sonido del medio ionizado.

Por otro lado, el flujo de fotones Ultravioleta provenientes de \thC{} a la distancia $D$ viene dado por:

\begin{align}
   F_* &= \frac{(1-f_d)\Nio_*}{4\piD^2} \label{eq:F-star} 
\end{align}

Donde $f_d$ es un factor relacionado con la absorción del polvo.

La condición de equilibrio de ionización implica que $F_{ph} = F_*$. Sin embargo, no todas las mediciones de las submuestras predicen una distancia $D$ que lleven a alcanzar dicho equilibrio. En la columna (6) de la tabla \ref{tab:arc-fits} mostramos el promedio de la planitud aparente de las sub-muestras que se aproximan más al equilibrio de ionización, y para las mediciones consecuentes de dicha tabla (columnas (7) en adelante) también se utilizan estas sub-muestras.

Por otro lado, la presión RAM en la cáscara viene dada por:

\begin{align}
  P_{in} = n_0\bar{m}M^2 \label{eq:P-in}
\end{align}

Donde $\bar{m} = 1.3 m_p$, $m_p = \SI{1.673e-24}{}$ es la masa del hidrógeno, $M$ es el número de Mach del flujo fotoevaporado antes del choque y $n_0$ es la densidad máxima del flujo fotoevaporado antes del choque, que se relaciona con $n_{\mathrm{IF}}$ como sigue:

\begin{align}
  n_0 = n_{\mathrm{IF}}\left(\frac{R_0}{r_{\mathrm{IF}}}\right)^{-2} M^{-1} \label{eq:density-scale}
\end{align}


Por último, la presión RAM del viento de \thC{} es:

\begin{align}
  P_* = \frac{\dot{M}^{0}_{w1}v_{w1}}{4\pi D^2} \label{eq:P-star}
\end{align}

Donde $\dot{M}^{0}_{w1} = \SI{3.5e-7}{M_\odot.yr^{-1}}$ es la tasa de pérdida de masa de \thC{}, y $v_{w1} = \SI{1400}{km.s^{-1}}$ es la velocidad terminal del viento de \thC{}.

De manera análoga al flujo, el equilibrio de presiones RAM que implica un choque estacionario se logra cuando $P_{in} = P_*$. %En la figura \ref{fig:wind-fits} se hace una comparación entre $F_{ph}$, $P_{in}$, $F_*$ y $P_{*}$ vs $D$ en un diagrama log-log para las sub-muestras que cumplen con la condición $\left|\frac{F_{ph}}{F_*}\right| < 10$. %Para los proplyds más lejanos se observa que las mediciones de las submuestras disponibles predicen una presión RAM mayor a la que se requiere para que el choque sea estacionario. Esto puede significar que por su distancia a \thC{} la presión del viento estelar deja de ser dominante y otra fuente también contribuye a confinar el choque de proa de estos proplyds. Esta hipótesis se está trabajando en un artículo por publicar.


\begin{figure}
  \centering
  \begin{tabular}{ccc}
    \includegraphics[width=0.3\linewidth]{./Figures/plot-wind-fits} & \includegraphics[width=0.3\linewidth]{./Figures/plot-wind-fits-beta} & \includegraphics[width=0.3\linewidth]{./Figures/plot-wind-fits-HA98} & \includegraphics[width=0.3\linewidth]{./Figures/plot-wind-fits-2} & \includegraphics[width=0.3\linewidth]{./Figures/plot-wind-fits-beta-2} & \includegraphics[width=0.3\linewidth]{./Figures/plot-wind-fits-HA98-2}
  \end{tabular}
  \caption{Diagrama log-log de Flujo y Presión RAM vs distancia, para las sub-muestras de cada proplyd, mostradas con colores, mientras que el flujo y presión RAM de la estrella se muestran con la línea gris. Las mediciones de la presión y el flujo utilizan las mediciones del radio del IF y densidad de la tabla \ref{tab:Prop-IF-par}, una tasa de fotones ionizan
    tes de $\Nio_* = \SI{1e49}{s^{-1}}$, que corresponde a una estrella de tipo espectral entre O5 y O6 (ver tabla \ref{tab:ionizing-radiation}), un factor de absorción por polvo  $f_d$ de 0.5 y el IF de los proplyds es de tipo D-crítico. En los páneles superiores el número de Mach del flujo fotoionizado es $M=3$, y en los páneles inferiores es $M=2$. En la columna izquierda se muestra el modelo A.: se utiliza la densidad y radio del Frente de Ionización de la tabla \ref{tab:Prop-IF-par}. En la columna central se muestra el modelo B.: la densidad se obtiene de a partir de la tasa de momentos $\beta$ obtenida para cada submuestra con la ecuación \ref{eq:b-density}. Y por último en la columna derecha se muestra el modelo HA, que utiliza las densidades de la tabla \ref{tab:Prop-IF-par} y además las inclinaciones reportadas en \citet{HA:1998}.}
  \label{fig:wind-fits}
\end{figure}

En la figura \ref{fig:wind-fits} mostramos diagramas log-log de flujo vs distancia y presión vs distancia de los proplyds (mostrados en colores) contra el flujo y presión del viento de \thC{} (mostrados con la línea gris). En todos los casos utilizamos el radio del Frente de Ionización reportado en \citet{HA:1998}, tabla 2, pero escalados a la distancia a ONC utilizada en este trabajo ($414 \pm \SI{6.8}{pc}$, \citet{Menten:2007}), y mostrados en la columna 3 de la tabla \ref{tab:Prop-IF-par}, para la densidad utilizamos dos modelos diferentes:

\begin{enumerate}[A.]
\item Utilizando los perfiles de brillo y la densidad máxima del Frente de Ionización obtenidos en \citet{HA:1998}, pero escalando por la distancia a \thC{} (columna 4 de la tabla \ref{tab:Prop-IF-par})
\item A partir de nuestras propias mediciones de la tasa de momentos $\beta$, y utilizando las ecuaciones (\ref{eq:beta-def}, \ref{eq:inner-dot-M}) la densidad se calcula como sigue:

  \begin{align}
    n_0 = \frac{\beta\left(\dot{M}^0_{w1}v_{w1}\right)\left(k + 1\right)}{2\pi R^2_0 \bar{m}\left(M a_{\mathrm{II}}\right)^2} \label{eq:b-density}
  \end{align}
  
\end{enumerate}

Y como modelo de referencia utilizamos las densidades de la columna 4 de la tabla \ref{tab:Prop-IF-par} y las inclinaciones reportadas en \citet{HA:1998} (modelo HA). Además, suponemos que el número de Mach del flujo fotoevaporado es de $M=[2, 3]$, que es un intervalo plausible para la velocidad del flujo fotoevaporado. Algunas observaciones cualitativas son las siguientes: la variación con el número de mach en la presión en los Modelos A. y HA es pequeña (la diferencia entre los diagramas con $M=3$ y $M=2$ es de $\log\left(\frac{P_{\mathrm{in}}(M=2)}{P_{\mathrm{in}}(M=3)}\right)\simeq -0.35$) mientras que en el modelo B. no existe variación en la presión pero sí con el flujo, a través de las ecuaciones (\ref{eq:F-ph}, \ref{eq:density-scale} y \ref{eq:b-density}). Otra observación importante es que en los modelos A. y HA no parece existir una correlación entre las presiones RAM de los proplyds con su distancia a \thC{}, aunque algunos proplyds se encuentran en equilibrio de presión con el viento de \thC{}. Por otro lado, en el modelo B sí parece existir una dependencia con distancia, similar a la del viento de \thC{}, pero más alta por un factor de $\sim 5$. Probablemente haga falta explorar en este modelo el comportamiento del factor de aborción por polvo $f_d$. Por último, los proplyds que tienen relacionada una distancia mayor a \thC{}, que son 177-341 y 180-331 no cumplen con equilibrio de presiones RAM en ningún modelo.

Los resultados de este capítulo serán publicados próximamente.