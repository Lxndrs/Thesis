\chapter{Conceptos fundamentales}
\label{sec:Modelo-generico}
Para este trabajo consideramos en general dos modelos de interacción de vientos:
\begin{itemize}
\item Una fuente localizada en el origen que emite un viento esférico que puede ser isotrópico o anisotrópico (figura \ref{fig:isotropic-aniso}) no acelerado que interactúa con el viento esférico isotrópico de otra fuente que se encuentra a una distancia $D$ de la primera(figura \ref{fig:crw-esquema}). A los choques de proa resultantes se conocen como ``Cantoides'' y ``Ancantoides'', respectivamente.
\item Una fuente localizada en el origen que emite un viento esférico isotrópico no acelerado que interactúa con un viento plano paralelo no acelerado y densidad constante (figura ). Los choques resultantes en este caso se conocen como ``Wilkinoides''.
\end{itemize}
El sitema en su conjunto tiene simetría cilíndrica.
\begin{figure}
  \includegraphics[width=0.5\linewidth]{./Figures/anisotropic-arrows}
  \caption{Representación esquemática de vientos con diferentes anisotropías: Arriba izquierda: Viento isotrópico esférico. Arriba derecha: viento isotrópico hemisférico. Abajo: Vientos anisotrópicos donde el parámetro $k$ indica el grado de anisotropía (ver capítulo \ref{chap:hipersonica})}
    \label{fig:isotropic-aniso}
\end{figure}
\begin{figure}
  \includegraphics[width=0.5\linewidth]{./Figures/bowshock-crw-variables}
  \caption{Representación esquemática del problema de interacción de dos vientos: Dos fuentes separadas por una distancia $D$ emiten un viento radial que forma un choque de proa a una distancia $R$ del origen. El sistema tiene geometría cilíndrica siendo el eje $z$ el eje de simetría. La forma del choque depende únicamente del ángulo polar $\theta$, medido a partir del origen. Otro ángulo que es de utilidad es $\theta_1$, que corresponde al ángulo polar medido a partir de la posición de la otra fuente.}
    \label{fig:crw-esquema}
\end{figure}

\section{Planitud y ``Alatud''}
\label{sec:char-rad}
Las cantidades medibles que nos ayudan a caracterizar un choque de proa las llamamos ``Radios característicos'' (ilustrados en la figura \ref{fig:char-radii}):
\begin{itemize}
\item Radio del choque en la dirección del eje de simetría del sistema. Denotado como $R_0$
\item Radio en dirección perpendicular al eje de simetría del sistema. Denotado como $R_{90}$
\item Radio de curvatura en la ``nariz'' del choque de proa. Denotado como $R_c$. En el apéndice \ref{app:math-curvature-radius} se muestra el procedimiento para obtener este radio para una curva genérica continua y derivable.
\end{itemize}

\begin{figure}
  \includegraphics[width=0.5\linewidth]{./Figures/characteristic-radii}
  \caption{Representación esquemática de los radios característicos de un choque de proa}
  \label{fig:char-radii}
\end{figure}
Un último parámetro es el ángulo asintótico de apertura de las alas, denotado como $\theta_\infty$. Sin embargo, esta medida solo aplica para choques cuyas alas son asintóticamente cónicas, y aún para éstos en la mayoría de los casos es dificil de medirlo debido a que el ángulo polar $\theta$ tiende al valor asintótico muy lentamente y además la emisión de las alas es bastante débil. Por otro lado, los radios característicos $(R_0, R_c, R_{90})$ son medibles observacionalmente en la mayoría de los casos. A partir de éstos, podemos determinar dos parámetros adimensionales llamados ``planitud'' y ``alatud''. El primero de éstos es una medida de qué tan plano es el choque de proa en la nariz o ``apex'', y lo denotamos con la letra griega $\Pi$, mientras que el segundo es una medida de qué tanto se abren las alas del choque de proa, y lo denotamos con la letra griega $\Lambda$. Ambos parámetros se definen a continuación:

\begin{align}
  \Pi \equiv \frac{R_c}{R_0} \label{eq:planitude}\\
  \Lambda \equiv \frac{R_{90}}{R_0} \label{eq:alatude}
\end{align}

\section{Proyección en el Plano del Cielo}
\label{sec:projection}

Para un choque de proa que es la vez geométricamente delgado y ópticamente delgado, únicamente se observa el borde de éste por
abrillantamiento al limbo, por lo tanto, su orientación respecto a la línea de visión modifica su forma respecto a la forma real del
choque. Para ello, rotamos el sistema de referencia del choque de proa en coordenadas cartesianas, denotado por $(x, y, z)$, por un ángulo que llamamos \textit{inclinación}, denotado por $i$, en el plano $xz$, de modo que la transformación entre el sistema de refencia del choque y el sistema de referencia del plano del cielo, denotado por
$(x', y', z')$ queda como sigue:
\begin{align}
  \left(
  \begin{array}{c}
    x' \\ y' \\ z'
  \end{array}
  \right) &=
  \left(
  \begin{array}{c}
    x\cos i - z\sin i \\ y' \\ z\cos i + x\sin i
  \end{array}
  \right)
  \label{eq:rotation}
\end{align}
Por otro lado, la forma tridimensional del choque de proa viene dado por:
\begin{align}
  \left(
  \begin{array}{c}
    x \\ y \\ z
  \end{array}
  \right) &=
            R(\theta)\left(
            \begin{array}{c}
              \cos\theta \\
              \sin\theta\cos\phi \\
              \sin\theta\sin\phi
            \end{array}
            \right)
\end{align}
La relación entre ambos sistemas de referencia se ilustra en la figura \ref{fig:reference}.
\begin{figure}
  \includegraphics[width=0.5\linewidth]{./Figures/projection-pos}
  \label{fig:reference}
  \caption{Sistema de referencia del choque vs sistema de referencia del plano del cielo. Los ejes $x'$ y $y'$ se encuentran en el plano del cielo, mientras el eje $z'$ es paralelo a la línea de visión. Solo la región del choque cuya tangente sea paralela a la línea de visión será visible por abrillantamiento al limbo.}
\end{figure}

\subsection{Vectores normal y tangente a la superficie}

Si definimos los vectores $\hat{n}$ y $\hat{t}$, como los vectores normal y tangente a la superficie, respectivamente para $\phi$ constante. En el caso $\phi = 0$ (figura \ref{fig:unit-vec}), ambos vectores se encuentran en el plano $xy$ y es fácil mostrar que:
\begin{align}
  \hat{t}_0 =
  \left(
  \begin{array}{c}
    -\cos\alpha \\
    \sin\alpha \\
    0
  \end{array}
  \right)
  \quad \mathrm{y} \quad
  \hat{n}_0 =
  \left(
  \begin{array}{c}
    \sin\alpha \\
    \cos\alpha \\
    0
  \end{array}
  \right)
  \label{eq:unit-vec}
\end{align}
Donde:
\begin{align}
  \tan\alpha = -\frac{dy}{dx} = \frac{1+\omega\tan\theta}{\tan\theta-\omega}
\end{align}
y:
\begin{align}
  \omega(\theta) = -\frac{1}{R}\frac{dR}{d\theta} 
\end{align}
Para otros valores de $\phi$, basta con hacer una rotación de las ecuaciones (\ref{eq:unit-vec}) alrededor del eje $x$. Para la conversión al sistema de referencia del plano del cielo se utiliza la ecuación (\ref{eq:rotation}):
\begin{align}
\begin{split}
  \hat{n}' &= \frac{1}{\left(1 + \omega^2\right)^{1/2}} \\
           & \times \left(
             \begin{array}{c}
               (\cos\theta+\omega\sin\theta)\cos i-(\sin\theta-\omega\cos\theta)\sin i\sin\phi\\
               (\sin\theta-\omega\cos\theta)\cos\phi \\
               (\cos\theta+\omega\sin\theta)\sin i+(\sin\theta-\omega\cos\theta)\sin\phi\cos i
             \end{array}
                    \right) \\
\end{split}\\
\begin{split}
    \hat{t}' &= \frac{1}{\left(1 + \omega^2\right)^{1/2}} \\
           & \times \left(
             \begin{array}{c}
               -(\sin\theta-\omega\cos\theta)\cos i-(\cos\theta+\omega\sin\theta)\sin i\sin\phi\\
               (\cos\theta+\omega\sin\theta)\cos\phi \\
               -(\cos\theta+\omega\sin\theta)\sin i+(\sin\theta-\omega\cos\theta)\sin\phi\cos i
             \end{array}
             \right)
\end{split} 
\end{align}


\begin{figure}
  \includegraphics[width=0.6\linewidth]{./Figures/bowshock-unit-vectors}
  \caption{Vectores unitarios normal y tangente a la superficie $R(\theta)$
    en un plano de azimuth $\phi$ constante.}
    \label{fig:unit-vec}
\end{figure}

\subsection{Línea tangente}
\label{sec:tangent-line}
Debido a que el choque es ópticamente delgado y geométricamente delgado, solo la región del choque cuya tangente sea paralela a la
línea de visión seré visible. Esto corresponde a una curva que denominamos \textit{línea tangente}, que debe cumplir con la siguiente
condición:
\begin{align}
  \hat{n}'\boldsymbol{\cdot} \hat{z}' = 0
\end{align}
Denotamos como $\phi_T$ al ángulo azimutal que cumple la condición anterior para una inclinación dada, en función del ángulo polar $\theta$:
\begin{align}
  \sin\phi_T = \tan i\tan\alpha = \tan i\frac{1+\omega\tan\theta}{\omega-\tan\theta}
  \label{eq:phi-tan}
\end{align}
De esta manera, la forma de la línea tangente del choque de proa, a la que llamamos \textit{forma proyectada} viene dada por:
\begin{align}
  \left(
  \begin{array}{c}
    x'_T \\
    y'_T \\
    z'_T
  \end{array}
  \right) =
  R(\theta)\left(
  \begin{array}{c}
    \cos\theta\cos i - \sin\theta\sin\phi_t\sin i \\
    \sin\theta\left(1-\sin^2\phi_T\right)^{1/2} \\
    \cos\theta\sin i + \sin\theta\sin\phi_T\cos i
  \end{array}
  \right) \label{eq:proj-shape}
\end{align}
En el caso general, $z'_T$ no es una función lineal de $x'_T$ y $y'_T$, por lo que la línea tangente no se encuentra en un plano. La forma aparente $(x'_T, y'_T)$  de la línea tangente también puede escribirse en coordenadas polares $(R', \theta')$, donde:
\begin{align}
  R'(\theta) = \left(x'^2_T + y'^2_T\right)^{1/2} & \mathrm{y} & \tan\theta' = \frac{y'_T}{x'_T}
  \label{eq:polar}
\end{align}
Es de notar a su vez que la ecuación (\ref{eq:phi-tan}) no tiene solución para valores arbitrarios de $\theta$ y de la inclinación, puesto que se requiere que $\left|\sin\phi_T\right| < 1$. Por tanto, la línea tangente solo existe para valores de $\theta$ tales que $\theta < \theta_0$ donde $\theta_0$ es el valor de $\theta$ en el eje de simetría de la línea tangente proyectada $(\theta'(\theta_0) = 0)$ y que se obtiene resolviendo la siguiente ecuación implícita:
\begin{align}
  \tan\theta_0 = \frac{|\tan i| + \omega(\theta_0)}{1 - \omega(\theta_0)|\tan i|}
  \label{eq:th-0}
\end{align}
Esto implica que si el choque de proa es suficientemente ``abierto'' $(\alpha > \alpha_{min})$, entonces para inclinaciones tales que
$|i| > 90^\circ - \alpha_{min}$ no existirá la línea tangente para ningún valor de $\theta$, es decir, el choque de proa se encontrará sufientemente ``de cara'' como para que ya no parezca un choque de proa para el observador.

\subsection{Radios característicos en el plano del cielo}

En orden de comparar la forma $R(\theta)$ con observaciones, es útil definir los radios característicos $R'_0$ y $R'_{90}$, donde $R'_0$ es el radio del eje de simetría aparente y $R'_{90}$ es el radio aparente en la dirección perpendicular a $R'_0$. Es decir
$R'_0 = x'_T(y'_t=0)$ y $R'_{90} = y'_t(x'_t = 0)$. Utilizando las ecuaciones (\ref{eq:phi-tan}) y (\ref{eq:proj-shape}) encontramos que:
\begin{align}
R'_0 = R(\theta_0)\cos(\theta_0 + i)
\label{eq:R0p}
\end{align}
Donde $\theta_0$ es la solución de la ecuación (\ref{eq:th-0}), y
\begin{align}
  R'_{90} = R(\theta_{90})\sin\theta_{90}\left(1-\sin^2\phi_T(\theta_{90})\right)^{1/2}
  \label{eq:R90p}
\end{align}
donde $\theta_{90}$ es la solución de la siguiente ecuación implícita:
\begin{align}
  \cot\theta_{90} = \frac{1 - \left(1+\omega(\theta_{90})^2\sin^22i\right)^{1/2}}
  {2\omega(\theta_{90})\cos^2i}
  \label{eq:th90}
\end{align}

\section{Cuádricas de Revolución}
\label{sec:quadrics}
\newcommand\Sin{\ensuremath{\mathcal{S}}}
\newcommand\Cos{\ensuremath{\mathcal{C}}}
\newcommand\Cot{\ensuremath{\mathcal{T}}}
\newcommand\Q{\ensuremath{\mathcal{Q}}}

En el caso general es difícil encontrar la forma aparente para un choque de proa siguiendo el formalismo desarrollado en la sección anterior, por lo que optamos por aproximar la forma éstos con una de las superficies más simples: las \textit{cuádricas de revolución}, que son superficies de revolución de las curvas cónicas. Dado el modelo general descrito en la \S \ref{sec:Modelo-generico}, haremos algunas restricciones para las superficies cuádricas que utilizaremos en este trabajo:
\begin{itemize}
  \item El eje focal se encuentra alineado con el eje $x$
  \item La posición del foco de la superficie cuádrica no necesariamente coincide con la posición de la fuente
  \item En el caso de las hipérbolas, solo tomamos una de las ramas de ésta.
\end{itemize}
Implementando dichas restricciones, utilizamos la representación paramétrica de las curvas cónicas en términos de un parámetro adimensional denotado con la letra $t$:
\begin{align}
  x &= x_0 + \sigma a\Cos(t) \\
  y &= b\Sin(t) 
\end{align}
Donde:
\begin{align}
  \Cos(t), \Sin(t) &=\left\lbrace
  \begin{array}{lr}
    \cos{t}, \sin t & \mathrm{elipses}\\
    \cosh{t}, \sinh{t} & \mathrm{hipérbolas}       
  \end{array}\right. \\
  \sigma &= \left\lbrace
  \begin{array}{lr}
    +1 & \mathrm{elipses} \\
    -1 & \mathrm{hipérbolas}
  \end{array}\right. \\
  x_0 &= R_0 -\sigma a \label{eq:x0} 
\end{align}
Donde $a$ y $b$ representan la longitud de los semi-ejes de la cónica en cuestión (Figura \ref{fig:conics}). $x_0$ representa la distancia entre el centro de la cónica y el origen. 

La forma polar del choque de proa $R(\theta)$ viene dada por:

\begin{align}
  \tan\theta &= \frac{b\Sin(t)}{a\Cos(t) + x_0} \label{eq:t-th-conversion} \\
  R &= \left(\left(a\Cos(t) + x_0\right)^2 + b^2\Sin^2(t)\right)^{1/2} 
\end{align}
El tipo de cónica lo podemos caracterizar mediante el parámetro $\Q$, donde:
\begin{align}
  \Q \equiv \sigma\frac{b^2}{a^2} \label{eq:conic-parameter-a-b}
\end{align}
Para las superficies abiertas (hiperboloides) tenemos que $\Q < 0$, mientras que para las superficies cerradas tenemos que $\Q > 0$. Casos particulares son la esfera $\Q = 1$ y el paraboloide $\Q = 0$. De manera equivalente se puede definir el ángulo $\theta_Q$ como sigue:
\begin{align}
  \tan\theta_Q = \sigma \frac{b}{a} \label{eq:thc}
\end{align}
Este ángulo se relaciona con la excentricidad de las cónicas (y que sustituye a esta última en este trabajo) como se muestra a continuación:
\begin{align}
  \tan\theta_Q = \sigma\sqrt{\left|1-e^2\right|}
\end{align}
\begin{figure}
  \begin{tabular}{cc}
    \includegraphics[width=0.4\linewidth]{./Figures/ellipse_edited} &
    \includegraphics[width=0.5\linewidth]{./Figures/hyperbola_edited}
  \end{tabular}
  \caption{Representación esquemática de: Izquierda: Elipse. Y, derecha: Hipérbola. En ambos casos se ilustran los parámetros relevantes de éstas y los radios característicos}
  \label{fig:conics}
\end{figure}

\begin{figure}
  \includegraphics[width = 0.5\linewidth]{./Figures/conic1}
  \caption{Familia de curvas cónicas, donde el valor del parámetro $\theta_Q$ varía desde $\theta_Q < 0$ (hipérbolas) hasta $\theta_Q > 0$ (elipses). Casos especiales son $\theta_Q = 0$ (parábola) y $\theta_Q = 45^\circ$ (círculo). Este parámetro sustituye en este trabajo a la excentricidad.}
  \label{fig:conics-family}
\end{figure}
El set de parámetros $(a, x_0, \Q)$ es suficiente para caracterizar a nuestras cuádricas de revolución: $\Q$ nos indica el tipo de cónica, $a$ establece la escala y $x_0$ el desplazamiento del centro a lo largo del eje x. Sin embargo, para futuras aplicaciones tanto a modelos de interacción de vientos como a observaciones (capítulos \ref{chap:hipersonica} y \ref{chap:proplyds}) nos sería util hacer la caracterización mediante los parámetros $(R_0, \Pi, \Lambda)$ (ver \S \ref{sec:char-rad}). Las equivalencias entre los dos sets de parámetros los calculamos a continuación:
\begin{align} 
  R_c &= \frac{b^2}{a} = a|\Q|\label{eq:R-curv-conic}\\
  R^2_{90} &= b^2\sigma\left(1 - \frac{x^2_0}{a^2}\right) = \Q\left(a^2 - x^2_0\right)\label{eq:R90-conic} 
\end{align}
Combinando las ecuaciones (\ref{eq:planitude}, \ref{eq:alatude}, \ref{eq:x0}, \ref{eq:conic-parameter-a-b}, \ref{eq:R-curv-conic}, \ref{eq:R90-conic}), obtenemos lo siguiente:

\begin{align}
  R_0 &= x_0 + \sigma a \\
  \Pi &= \frac{a|\Q|}{x_0 + \sigma a} = \frac{a\Q}{\sigma\left(x_0 + \sigma a\right)} = \frac{a\Q}{\left(a + \sigma x_0\right)}\\
  \Lambda &= \left(\Q\frac{a-\sigma x_0}{a + \sigma x_0}\right)^{1/2}
\end{align}

De aquí podemos escribir el parámetro de las cuádricas $\Q$ en términos de la planitud y la alatud:

\begin{align}
  \Q = 2\Pi - \lambda^2 \label{eq:quadric-parameter-pi-lambda}
\end{align}

Por tanto, el signo de $2\Pi - \lambda^2$ determina si una cuádrica es esferoidal o hiperboloidal. En la figura \ref{fig:conics-family} mostramos como, para planitud constante, podemos tener una familia de cónicas variando únicamente la alatud, y por consiguiente, el parámetro $\Q$.
\subsection{Proyección en el plano del cielo}

El objetivo de esta sección es obtener la forma proyectada de las cuádricas de revolución, puesto que son una aproximación buena y mucho más sencilla a la forma real de un choque de proa. La forma tridimensional de las cuádricas de revolución viene dada por:

\begin{align}
  x &= a\Cos(t) + x_0 \\
  y &= b\Sin(t)\cos\phi \\
  z &= b\Sin(t)\sin\phi
\end{align}

Siguiendo el procedimiento mostrado en la \S \ref{sec:projection} calculamos el ángulo azimutal $\phi$ que cumple con el criterio de ser tangente al la línea de visión:

\begin{align}
  \sin\phi_T = \frac{b}{a}\tan i\Cot(t) 
\end{align}
Donde:
\begin{align}
  \Cot(t) = \left\lbrace
  \begin{array}{lr}
    \cot t & \mathrm{si}~\theta_c > 0 \\
    \coth t & \mathrm{si}~\theta_c < 0 
  \end{array}
  \right.
\end{align}
Ahora utilizamos la ecuación (\ref{eq:rotation}) para obtener la forma aparente de una cuádrica dada:
\begin{align}
  x'_T &= \frac{\Cos(t)}{a\cos i}\left(a^2\cos^2 i \pm b^2\sin^2 i\right) + x_0\cos i
  \label{eq:x-prime-proj}\\
  y'_T &= b\Sin(t)\left(1 - \frac{b^2}{a^2}\tan^2 i\Cot^2(t)\right)^{1/2}
  \label{eq:y-prime-proj}
\end{align}
Se espera que la forma proyectada de una cuádrica dada sea otra cuádrica del mismo tipo, por lo que es posible escribir las ecuaciones (\ref{eq:x-prime-proj}) y (\ref{eq:y-prime-proj}) de la siguiente manera: 
\begin{align}
  x'_T &= a'\Cos(t') + x'_0 \label{eq:xtprime}\\
  y'_T &= b'\Sin(t') \label{eq:ytprime}
\end{align}
Donde:
\begin{align}
  x'_0 &= x_0\cos i \\
  a' &= \left(a^2\cos^2 i \pm \b^2\sin^2 i\right)^{1/2} \label{eq:a-prime}\\
  b' &= b \label{eq:b-prime}\\
  \Cos(t') &= \frac{a'\Cos(t)}{a\cos i} \\
  \Sin(t') &= \left(1 - \Cos^2(t')\right)^{1/2}
\end{align}
Dos cantidades que nos van a ser de utilidad son los valores del parámetro $t$ que denominaremos $t_0$ y $t_{90}$ y son tales que $t'(t_0) = 0$ y $t'(t_{90}) = \frac{\pi}{2}$ o bien $y'_T(t_0) = 0$ y $x'_T(t_{90}) = 0$. De esta manera obtenemos las siguientes ecuaciones implícitas evaluando las ecuaciones (\ref{eq:x-prime-proj}) y(\ref{eq:y-prime-proj}) en $t=t_{90}$ y $t=t_0$ respectivamente:
\begin{align}
  \Cot(t_0) &= \frac{a}{b}\cot{i} = \frac{\cot{i}}{\left|\tan\theta_c\right|} \label{eq:t0}\\
  \Cos(t_{90}) &= -\frac{ax_0\cos^2{i}}{a^2\cos^2{i}\pm b^2\sin^2{i}} \label{eq:t90}
\end{align}
Los radios característicos aparentes los podemos calcular a partir de las ecuaciones (\ref{eq:xtprime}) y (\ref{eq:ytprime}) como se hizo para los radios característicos en el sistema no primado:
\begin{align}
  R'_0 &= \pm a' + x'_0\\
  R'_c &= \frac{b'^2}{a'}\\
  \tan\theta'_c &= \pm\frac{b'}{a'} \\
  R'_{90} &= \left(2R'_c \mp \tan^2\theta'_c\right)^{1/2}
\end{align}
Utilizando las ecuaciones (\ref{eq:x0}), (\ref{eq:a-prime}) y (\ref{eq:b-prime}), utlizando la definición $D' = D\cos i$ e introduciendo la función $f(i;\theta_c)\equiv \left(1 \pm \tan^2\theta_c\tan^2i\right)^{1/2}$ obtenemos ecuaciones explícitas para los radios característicos en el sistema de referencia del plano del cielo en términos de la inclinación:
\begin{align}
  \frac{q'}{q} &= 1 \pm \tilde{R}_c\cot^2\theta_c\left(f(i;\theta_c) - 1\right) \\
  \tilde{R}'_c &= \frac{\tilde{R_c}}{\cos^2if(i;\theta_c)\frac{q'}{q}} \label{eq:Rpc-quad}\\
  \tan\theta'_c &= \frac{\tan\theta_c}{\cos if(i;\theta_c)} \label{eq:thcp-quad}\\
  \tilde{R}'_{90} &= \left(\frac{2\tilde{R}_cf(i;\theta_c) \mp
                    \tan^2\theta_c\frac{q'}{q}}{q'/q}\right)^{1/2}\frac{\sec i}{f(i;\theta_c)}
  \label{eq:Rp90-quad}
\end{align}
Cuando $\tilde{R}'_{90}$ es medible, entonces es posible hacer diagramas de diagnóstico como
el de la figura \ref{fig:diagnostic} para comparar con observaciones, independientemente de cualquier modelo de choques de proa.
\begin{figure}
  \includegraphics[width=0.5\linewidth]{./Figures/projected-R90-vs-Rc}
  \caption{Diagrama de diagnóstico $\tilde{R}'_{90}$ vs $\tilde{R}'_c$ para las cuádricas de revolución. En la región sin sombrear se representan las superficies abiertas (hiperboloides, $\theta_c <0$), mientras que la región más oscura representa a elipsoides prolatos  $(0 < \theta_c < 45^\circ)$ y la región poco sombreada a elipsoides oblatos $(\theta_c > 45^\circ)$}
  \label{fig:diagnostic}
\end{figure}
