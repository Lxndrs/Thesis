\label{chap:hipersonica}

El problema de interacción de dos vientos es de gran interés en astrofísica, y
ha sido estudiado en múltiples ocasiones, principalmente mediante simulaciones
hidrodinámicas. Sin embargo, cuando se toman en cuenta diversos factores, incluídos
conservación de masa, momento y momento angular, el problema puede resolverse de manera
algebraica.
\section{Cantidades conservadas en un flujo hipersónico de capa delgada}

Consideramos dos flujos hipersónicos, no acelerados que forman una capa estacionaria delgada
formada por dos choques radiativos separados por una discontinuidad de contacto. El sistema
tiene geometría cilíndrica y los vientos no tienen velocidad azimutal. Bajo estos términos,
describimos la posición de la capa delgada como $R(\theta)$, donde $R$ es el radio de la capa
medido a partir de la posición del origen del viento con menor momento y $\theta$ es el ángulo
polar. Si asumimos que el gas chocado está bien mezclado, entonces tiene una sola velocidad
pos choque dada por:

\begin{align}
  \vec{v} = v_r \hat{r} + v_z \hat{z}
\end{align}

Donde el eje de simetría del sistema es paralelo a $\hat{z}$, y $\hat{r}$ es el radio cilíndrico.
Definimos $\dot{M}(\theta)$, $\vec{\dot{\Pi}}(\theta)$ y $\dot{J}(\theta)$ como la tasa de pérdida
de masa, la tasa de momento y la tasa de momento angular, respectivamente, de la capa delgada
integradas desde $\theta=0$ hasta $\theta$. Éstas se calculan de la siguiente manera:

\begin{align}
  \vec{\dot{\Pi}}(\theta) &= \dot{\Pi}_r(\theta) \hat{r} + \dot{\Pi}_z(\theta) \hat{z} = \dot{M}\left(
                      v_r \hat{r} + v_z\hat{z}\right) \label{eq:dot-pi}\\
  \vec{\dot{J}}(\theta) &= \vec{R}(\theta) \times \vec{\dot{\Pi}}(\theta)  \\
  \dot{M}(\theta) &= \dot{M}_w(\theta) + \dot{M}_{w1} \label{eq:dot-M}
\end{align}
Donde $\vec{R}(\theta)\equiv R(\theta)\sin\theta \hat{r} + R\cos\theta \hat{z}$. Resolviendo el producto
cruz y tomando su magnitud encontramos que:

\begin{align}
  \dot{J}(\theta) &= \dot{M}(\theta)R(\theta)v_\theta \label{eq:dot-J}\\
  \mathrm{donde:~} & v_\theta = v_r\cos\theta - v_z\sin\theta \label{eq:v-theta}
\end{align}

Por otro lado, al asumir estado estacionario, necesitamos que la tasa de pérdida de masa, la tasa de
momento y la tasa de momento angular de la capa delgada sean iguales a aquellas inyectadas por los dos
vientos. Entonces definimos estas cantidades como $\dot{M}_w$, $\dot{\Pi}_{wr}$, $\dot{\Pi}_{wz}$ y
$\dot{J}_{w}$ para el viento con menor momento, y para el otro viento se utiliza la misma notación
solo que utilizando el subíndice ``w1''. De esta forma tenemos que:
\begin{align}
  \dot{\Pi}_r(\theta)\hat{r} + \dot{\Pi}_z(\theta)\hat{z} &= \left[\dot{\Pi}_{wr}(\theta)+ \dot{\Pi}_{wr1}(\theta)
                                                            \right]\hat{r} + \left[\dot{\Pi}_{wz}(\theta)+ \dot{\Pi}_{wz1}(\theta)\right]\hat{z}
                                                            \label{eq:Pi-2} \\
  \dot{J} &=\dot{J}_w(\theta) + \dot{J}_{w1}(\theta) \label{eq:J-2}\\
  \dot{M}(\theta) &= \dot{M}_w(\theta) + \dot{M}_{w1}(\theta) \label{eq:M-2}
\end{align}

Combinando las ecuaciones (\ref{eq:dot-pi}), (\ref{eq:dot-J}), (\ref{eq:dot-M}), (\ref{eq:Pi-2}), (\ref{eq:J-2}) y (\ref{eq:M-2})
encontramos que:

\begin{align}
  \dot{M}(\theta)\left[v_r \hat{r} + v_z\hat{z}\right] &= \left(\dot{\Pi}_{wr}(\theta) + \dot{\Pi}_{wr1}(\theta)\right)\hat{r} +
                                                         \left(\dot{\Pi}_{wz}(\theta) + \dot{\Pi}_{wz1}(\theta)\right)\hat{z} \\
  \dot{M}(\theta)v_\theta R(\theta) &= \dot{J}_w(\theta) + \dot(J)_{w1}(\theta)
\end{align}
Y finalmente combinando con la ecuación (\ref{eq:v-theta}) resolvemos para $R(\theta)$:
\begin{align}
  R(\theta) = \frac{\dot{J}_w(\theta) + \dot(J)_{w1}(\theta)}{\left(\dot{\Pi}_{wr}(\theta) + \dot{\Pi}_{wr1}(\theta)\right)\cos\theta
  - \left(\dot{\Pi}_{wz}(\theta) + \dot{\Pi}_{wz1}(\theta)\right)\sin\theta} \label{eq:R-wind}
\end{align}



\section{Problema de Interacción de Dos Vientos}

Aplicamos el formalismo ya mencionado para la interacción de dos vientos radiales. El viento con menor momento
se localiza en el origen, y su densidad a radio fijo varía con el ángulo polar como una ley de potencias
(figura \ref{fig:isotropic-aniso}):
\begin{align}
  n(\theta) = n_0\cos^k\theta \label{eq:anisotropic-density}
\end{align}
Donde el índice $k$ indica el grado de anisotropía del viento ``interno''. Algunos casos particularmente interesantes
son el viento para un proplyd \citep{HA:1998}, donde $(k=1/2)$ y el caso ``isotrópico'' \citep{Canto:1996} donde $k=0$.
Por el momento restringimos al viento ``externo'' como isotrópico. El problema se muestra de manera esquemática en la
figura \ref{fig:crw-esquema}.

Utilizando la ecuación (\ref{eq:anisotropic-density}) encontramos que la tasa de pérdida de masa está dada por:

\begin{align}
  \dot{M}_w = \int^\theta_0\int^{2\pi}_0\rho_w v_w~d\theta~d\phi =
  \frac{M^0_w}{2\left(k+1\right)}\left(1 - \cos^{k+1}\theta\right)
\end{align}
Donde $v_w$ es la velocidad del viento inteno, $\rho_w = n\bar{m}$  es su densidad, $n$ se obtiene de la ecuación
(\ref{eq:anisotropic-density}), $M^0_w = 4\pi r^2_0v_w n_0 \bar{m}$ es la tasa de pérdida de masa integrada hasta
$\theta = \pi$ para el caso isotrópico, $\bar{m}$  es la masa promedio de las partículas del viento y
$r_0$ es el radio del viento al cual se alcanza la velocidad terminal $v_w$. Para un proplyd consideramos que dicho
radio es el del frente de ionización.

Con esto, obtenemos las tasas de momento y momento angular:
\begin{align}
  \dot{\Pi}_{wz} &= \int^\theta_0 v_w\cos\theta~d\dot{M}_w =
                   \frac{v_w \dot{M}^0_w}{2\left(k+2\right)}\left(1 - \cos^{k+2}\theta\right) \label{eq:Pi-wz} \\
  \dot{\Pi}_{wr} &= \int^\theta_0 v_w\sin\theta~d\dot{M}_w = \frac{1}{2}\dot{M}^0_w v_w I_k(\theta) \\
  \dot{J}_w &= \int^\theta_0 |\vec{R} \times \vec{v}_w|d\dot{M}_w = 0
\end{align}

Donde la integral $I_k(\theta) = \int^\theta_0 \cos^k\theta \sin^2\theta~d\theta$ tiene solución analítica para $k=0$,
es una integral elíptica de segundo tipo cuando $k=\frac{1}{2}$ y su solución es aun más compleja para el resto de los
casos. Las tasa de momento angular para el viento interior es cero debido a que éste se mide respecto al origen, donde
se localiza la fuente con menor momento. En este punto los vectores de posición y velocidad para un valor de $\theta$
dado son paralelos.

Para el viento exterior consideramos dos casos principales: un viento esférico e isotrópico y un viento
plano--paralelo de densidad y velocidad constante.

\subsection{Interacción con un viento esférico isotrópico}

En este caso tomamos como variable independiente al ángulo polar medido a partir de la posición de la fuente del viento
externo, denotado por $\theta_1$. De esta forma las tasas de pérdida de masa, momento y momento angular quedan como sigue:

\begin{align}
  \dot{M}_{w1} &= \frac{M^0_{w1}}{2}\left(1 - \cos\theta_1\right)\\
  \dot{\Pi}_{wz1} &= -\frac{v_{w1}\dot{M}^0_{w1}}{4}\sin^2\theta_1\\
  \dot{\Pi}_{wr1} &= \frac{v_{w1}\dot{M}^0_{w1}}{4}\left(\theta_1 - \sin\theta_1\cos\theta_1\right)\\
  \dot{J}_{w1} &= \int^{\theta_1}_0 R(\theta)v_{w1}\sin(\pi-\theta-\theta_1)~d\dot{M}_{w1} \label{eq:J1}
\end{align}

Utilizando la ley de los senos (ver figura \ref{fig:crw-esquema}), ecuación (\ref{eq:J1}) queda como sigue:

\begin{align}
  \dot{J}_{w1} &= Dv_{w1}\int^{\theta_1}_0 \sin\theta_1~d\dot{M}_{w1} =
                 \frac{v_{w1}\dot{M}^0_{w1}}{4}\left(\theta_1 - \sin\theta_1\cos\theta_1\right) D \label{eq:J1-iso}
\end{align}

Por otro lado, de la figura \ref{fig:crw-esquema}, podemos deducir la siguiente relación geométrica entre $R(\theta)$,
$\theta$ y $\theta_1$:
\begin{align}
  \frac{R(\theta)}{D} &= \frac{\sin\theta_1}{\sin(\theta+\theta_1)} \label{eq:R-geometric}
\end{align}

Combinando las ecuaciones (\ref{eq:R-wind}), (\ref{eq:Pi-wz}) - (\ref{eq:J1-iso}) y (\ref{eq:R-geometric}) obtenemos una ecuación
implícita que nos indica la dependencia de $\theta_1$ con $\theta$:

\begin{align}
  \theta_1\cot\theta_1 -1 = 2\beta I_k(\theta)\cot\theta - \frac{2\beta}{k+2}\left(1 - \cos^{k+2}\theta\right) \label{eq:th1-th} 
\end{align}
Donde $\beta = \frac{\dot{M}^0_w v_w}{\dot{M}^0_{w1}v_{w1}}$ es el cociente del momentos entre los vientos. Este parámetro, junto con el
índice de anisotropía $k$ son los que determinan la forma del choque de proa. 

Los radios característicos en este caso se muestran a continuación. El procedimiento detallado se puede consultar en el apéndice
\ref{app-derivation-radii}:

\begin{align}
  \frac{R_0}{D} = \frac{\beta^{1/2}}{1+\beta^{1/2}} \\
    \tilde{R}_{90} &= \frac{\left(3\xi\right)^{1/2}\left(1+\beta^{1/2}\right)}
                     {\left(1+\frac{1}{5}\xi\beta\right)^{1/2}\left(1-\xi\beta\right)} \\
  \tilde{R}_c = \left(1 - 2\gamma\right)^{-1} \\
  \mathrm{Donde:~} \gamma &= \frac{C_{k\beta}}{1+\beta^{1/2}} + \frac{1 + 2\beta^{1/2}}{6} 

\end{align}


\subsection{Interacción con un viento plano--paralelo}

En este caso las tasas de pérdida de masa, de momento y momento angular del viento plano--paralelo quedan como sigue:

\begin{align}
\end{align}
