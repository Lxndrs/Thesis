\section{La Nebulosa de Orión}
\section{Estrellas ``Errantes''}
\section{Discos Protoplanetarios}
\section{Proplyds}
\subsection{Descubrimiento}
Observaciones en óptico de la región del trapecio en filtros de banda
angosta de diferentes líneas de emisión tales como $H\alpha$, $H\beta$,
$[OIII]$, $[NII]$, $[SII]$ y continuo, revelaron la existencia de
objetos puntuales únicamente visibles en líneas de alta ionización
($H\alpha$, $H\beta$ y $[OIII]$) que fueron inicialmente denominados como
``condensaciones nebulares'' \citep{Laques:1979}. 

Hasta el momento no se sabía con certeza si ``condensaciones nebulares''
eran en realidad condensaciones nebulares (regiones donde la densidad de
la nebulosa es inusualmente alta por alguna razón o bien esferas de gas
molecular cuya envolvente fue ionizada y que la radiación de la estrella
central la está ``erosionando'') o si se trataba de protoestrellas
de baja masa cuyo disco protoplanetario estaba siendo fotoevaporado por
la estrella central \citep{churchwell:1987}. No fue sino hasta que se contó
con observaciones de alta resolución con el Telescopio Espacial Hubble (HST)
que se se pudo determinar la verdadera naturaleza de estos objetos
\citep(ODell:1993) y la razón por la que se les denominó ``proplyds''
(PROtoPLanetarY DiskS). A su vez se encontraron por primera vez arcos
delgados y otras estructuras de gran interés.

\subsection{Mecanismos de fotoevaporación}



\section{Objetos LL}
\subsection{Mapa de Objetos}