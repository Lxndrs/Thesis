\documentclass{beamer}
\usetheme{Madrid}
%\mode<presentation>
%{
%\usetheme{Amsterdam}

%\setbeamercovered{transparent}
%}

\usepackage[utf8]{inputenc}
%\usepackage{default}
\usepackage{graphicx}
\usepackage{hyperref}
\usepackage{color}
\usepackage{natbib}
\usepackage{aastex-compat}
\title[PhD Thesis]{Estudio de la Interacción de Flujos Múltiples de Fuentes Astrofísicas, Aplicada a los Proplyds Clásicos de la Nebulosa de Orión}
\author[M.C Jorge Alejandro Tarango Yong]{Presents: M.C Jorge Alejandro Tarango Yong \\
  PhD Advisor: Dr. William Henney \\
  Tutorial Committee: Dr. Javier Ballesteros, Dr. Luis Zapata}
\date{\today}
\definecolor{AZUL}{HTML}{000032}
\definecolor{DORADO}{HTML}{A87A00}
\setbeamercolor{title}{fg=AZUL, bg=DORADO}
\setbeamercolor{frametitle}{fg=AZUL, bg=DORADO}
\setbeamercolor{structure}{fg=AZUL, bg=DORADO}
\setbeamertemplate{enumerate items}[square]
\setbeamertemplate{section in toc}[square]
\begin{document}
\frame{\frametitle{PhD Thesis Defense}
  \includegraphics[scale=0.1]{../Figures/logo}
  \maketitle}

\frame{\frametitle{Index}
  \tableofcontents}

\section{Orion Nebula}
\frame{1}
\section{Bowshocks in the ISM}
\frame{2}
\section{Fundamental Concepts}
\frame{3}
\section{Thin Shell Model}
\frame{4}
\section{Results Obtained to the Classical Proplyds of Orion Nebula}
\frame{5}
\section{Summary and Conclusions}
\frame{6}
\end{document}

